\documentclass[12pt]{article}
\usepackage
[pdftex,pagebackref,letterpaper=true,colorlinks=true,pdfpagemode=none,urlcolor=blue,linkcolor=blue,citecolor=blue,pdfstartview=FitH]{hyperref}

\usepackage{amsmath,amsfonts}
\usepackage{graphicx}

\bibliographystyle{alpha}

\setlength{\oddsidemargin}{0pt}
\setlength{\evensidemargin}{0pt}
\setlength{\textwidth}{6.0in}
\setlength{\topmargin}{0in}
\setlength{\textheight}{8.5in}

\setlength{\parindent}{0in}
\setlength{\parskip}{5pt}


%%%%%%%%% For wordpress conversion

\def\more{}

\usepackage{ulem}
\def\em{\it}
\let\emph=\it

%%%%%%%%% For WordPress conversion

\newif\ifblog
\newif\iftex
\blogfalse
\textrue

\def\more{}


\usepackage{ulem}
\def\em{\it}
\def\emph#1{\textit{#1}}

\def\image#1#2#3{\begin{center}\includegraphics[#1pt]{#3}\end{center}}

\let\hrefnosnap=\href

\newenvironment{btabular}[1]{\begin{tabular} {#1}}{\end{tabular}}

\newenvironment{red}{\color{red}}{}
\newenvironment{green}{\color{green}}{}
\newenvironment{blue}{\color{blue}}{}



%%%%%%%%% Complexity classes

\def\np{{\rm NP}}
\def\p{{\rm P}}
\def\dtime{{\rm DTIME}}
\def\ntime{{\rm NTIME}}


%%%%%%%%% Typesetting shortcuts

\def\B{\{0,1\}}
\def\xor{\oplus}

\def\P{\mathop{\mathbb P}}
\def\E{\mathop{\mathbb E}}
\def\var{{\bf Var}}

\def\N{{\mathbb N}}
\def\Z{{\mathbb Z}}
\def\R{{\mathbb R}}
\def\G{{\mathbb G}}


\def\bz{{\bf z}}

\def\true{{\tt true}}
\def\false{{\tt false}}

%%%%%%%%% Theorems and proofs

\newtheorem{exercise}{Exercise}
\newtheorem{theorem}{Theorem}
\newtheorem{lemma}[theorem]{Lemma}
\newtheorem{definition}[theorem]{Definition}
\newtheorem{example}[theorem]{Example}
\newtheorem{remark}[theorem]{Remark}
\newenvironment{proof}{\noindent {\sc Proof:}}{$\Box$ \medskip} 


%%%%%%%%% To typeset the header

\def\courseprof{Luca Trevisan}
\def\coursenum{CS276}
\def\coursename{Cryptography}

\newlength{\tpush}
\setlength{\tpush}{2\headheight}
\addtolength{\tpush}{\headsep}


\newcommand{\handout}[3]{\noindent\vspace*{-\tpush}\newline\parbox{\textwidth}
{U.C. Berkeley --- \coursenum : \coursename \hfill Handout #1 \newline
\courseprof \hfill #2 \newline
\mbox{}\hrulefill\mbox{}}\vspace*{1ex}\mbox{}\newline
\bigskip
\begin{center}{\Large\bf  #3}\end{center}
\bigskip}

\newcommand{\classnotes}[2]{\handout{N#1}{#2}{Notes for Lecture #1}}

\newcommand{\problemset}[2]{\noindent\vspace*{-\tpush}\newline\parbox{\textwidth}
{U.C. Berkeley --- \coursenum : \coursename \hfill Handout #1 \newline
\courseprof \hfill #2 \newline
\mbox{}\hrulefill\mbox{}}\vspace*{1ex}\mbox{}\newline
\bigskip
\bigskip}




\begin{document}


\problemset{MT}{March 31, 2009}

{\em Last updated April 1, 2009}

\section*{Midterm}

\bigskip

This midterm exam is due by email to {\sf luca@cs.berkeley.edu}
at noon, on Thursday, April 9, 2009. Put the word ``midterm'' in
the subject line.

Some notes that apply to various problems:
\begin{itemize}
\item We use $||$ to denote string concatenation. For
example if $a= 011$ and $b= 101$, then $a||b = 011101$.
\item If $a$ and $b$ are bit strings, then $a+b$ is the bitwise xor
of $a$ and $b$. For example if $a=011$ and $b=101$, then
$a+b = 110$.
\item About the use of $O()$ notation:
When we ask to prove that ``If $X$ is $(t,\epsilon)$-secure
then $Y$ is $(t-O(r),\epsilon)$-secure,'' we mean ``Prove that
there is a $c$ such that, for all $t\geq 2$, $r\geq 2$ and $0<\epsilon \leq 1$,
it holds that if $X$ is $(t,\epsilon)$-secure then $Y$ is $(t-c\cdot r,\epsilon)$-secure.''
That is, the constant in the $O()$ notation must be independent of all
other parameters $t,r,\epsilon$, and be entirely determined by the
details of your reduction and the specific definition of the model of
computation. (That is, what precisely can be computed in one step of computation.)

\item Solve the problems individually, do not collaborate with your
classmates or others.

\item If in doubt, do not hesitate to ask for clarifications.
Maybe  that there is a typo that make a problem
unsolvable as stated.

\end{itemize}

Solve the following problems:

\begin{enumerate}

\item {\bf Using a pseudorandom generator to reduce key length [20 points]}

Let $G : \B^{256} \to \B^{1024}$ be a $(2^{90},2^{-40})$-secure
pseudorandom generator, and $F: \B^{1024} \times \B^{128} \to \B^{128}$
be a $(2^{80},2^{-40})$-secure pseudorandom function which
is computable with complexity $\leq 1,000,000$. (That is,
$F$ has a $1024$-bit key, a $128$-bit input and a $128$-bit output.)

Consider the function $F' : \B^{256} \times \B^{128} \to \B^{128}$
defined as

\[ F'_K(x) := F_{G(K)} (x) \]

Prove that $F'$ is a $(2^{70},2^{-39})$-secure pseudorandom function.

\item {\bf Encryption using a Pseudorandom Permutation [30 points].}

Suppose $F: \B^k \times \B^m \to \B^m$ is a 
$(t,\epsilon)$-secure pseudorandom permutation computable in time $\leq r$
and $I: \B^k \times \B^m \to \B^m$ is its inverse, also computable
in time $\leq r$.

Consider the following scheme encryption scheme
$(E,D)$ to encrypt blocks of $m/2$ bits.

\begin{itemize}

\item $E(K,M)$: pick a random $R\in \B^{m/2}$, output $F(K,(M||R))$.

\item $D(K,C)$: compute $M' := I(K,C)$, output the first $m/2$ bits of $M'$

\end{itemize}

Show that $(E,D)$ is $(t/O(r), 2\epsilon + 2\cdot t \cdot 2^{-m/2})$ CCA-secure.


\item {\bf MACs from cryptographic hash functions. [15 points]}

Suppose that $H: \B^k \times \B^{2m} \to \B^m$ is a family
of $(t,\epsilon)$-secure collision-resistant hash functions.
Consider the following candidate MAC scheme for
signing $m$-bit messages:

\begin{itemize}
\item Secret key: a random $m$-bit string $K$;
\item $Tag(K,M):$ pick a random $s\in \B^k$, and output
the tag $s || H_s (K||M)$

\item $Verify(K,M,(s||h))$ check that $h=H_s(K||M)$
\end{itemize}

Show that, just from the assumption that $H$ is $(t,\epsilon)$-secure,
it is not possible to deduce the security of the above scheme.

In more detail: show that if a family of $(t,\epsilon)$-secure collision-resistant
hash functions $H: \B^k \times \B^{2m} \to \B^{m-1}$ exists, then
there is a family $H': \B^k \times \B^{2m} \to \B^{m}$ that is 
still $(t,\epsilon)$-secure  but is such that if $H'$ is used in the above
construction, then there is an attack that runs in linear time $O(k+m)$
and produces a forged tag with probability 1 after seeing only
one message-tag pair.

\iffalse

\item {\bf How not to construct a block cipher. [14 points]}

Consider the following simplified version of one round of AES.

The construction has a parameter $k$ (for example $k=128$) and,
for every fixed $k$ we define a function $AES1^{(k)} : \B^{2k + \log_2 k}
\times \B^k \to \B^k$. 

(For example, if $k=128$, $AES1^{(128)}$ has
a key of length $263$, and for every fixed key $K\in \B^{263}$,
$AES1^{(128)}_K(\cdot)$ is a bijection mapping 128 bits into 128 bits.)

We think of the key as being of the form
$(r_1,r_2,i)$, where $r_1,r_2$ are $k$-bit strings, and $i$
is an integer in the range $\{ 0,\ldots, k-1\}$. 

The following specification refers to a fixed bijective function $f: \B^{k/8} \to \B^{k/8}$ (for
example, inverse in $GF(2^{k/8})$). The choice of $f$ does not
affect the solution of this problem.

We define $AES1^{(k)}$ as follows.


$AES1^{(k)}_{(r_1,r_2,i)}  (x)$:
\begin{itemize}

\item $y_1 := x + r_1$
\item $y_2 := $cyclic$\_$shift$(y_1,i)$

// That is, set $y_2[j] := y_1[j-i \bmod k]$ for all $j\in \{0,\ldots,k-1\}$
\item divide $y_2$ into $8$ blocks each of $k/8$ bits; Let $y_3$ be the 
string obtained by applying $f()$ to each block and then concatenating
the outputs
\item $y_4 := y_3 + r_2$
\item output $y_4$

\end{itemize}

Prove that, regardless of the choice of $f()$, this construction
is not a good pseudorandom permutation; specifically, show that
there is an attack that runs in time $O(k)$ and, for large enough $k$,
has distinguishing probability $\geq 1/2$. (In fact, you should be
able to get distinguishing probability larger than $.999999$
for $k=128$, and $\geq 1-exp(-\Omega(k))$ asymptotically.)

[Hint: what happens when you evaluate $AES1^{(k)}_{(r_1,r_2,i)}(\cdot)$
on two inputs that don't differ much?]
\fi

\item {\bf One-way functions and function composition. [20 points]} 

Let $f: \B^n \to \B^n$ a $(t,\epsilon)$ one-way function
and $g:\B^n \to \B^n$ be a bijection computable in time $\leq r$.

\begin{enumerate}

\item
Show that $h_1 (x) := g(f(x))$ is $(t-r,\epsilon)$-one way.

\item Show that $h_2 (x) := f(g(x))$ is $(t-r,\epsilon)$-one way.


\end{enumerate}


\item {\bf Composition of Pseudorandom Generators. [15 points]}

Suppose that $H,G : \B^n \to \B^{2n}$ are $(t,\epsilon)$-secure
pseudorandom generators computable in time $\leq r$.

Prove that the mapping $B: \B^{2n} \to \B^{4n}$ defined as

\[ B(x,y) := G(x) || H(y) \]

is a $(t- O(r+n), 2\epsilon)$-secure pseudorandom generator.



\end{enumerate}
\end{document}
