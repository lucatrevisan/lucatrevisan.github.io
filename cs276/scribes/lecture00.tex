\documentclass[11pt]{article}
\usepackage
[pdftex,pagebackref,letterpaper=true,colorlinks=true,pdfpagemode=none,urlcolor=blue,linkcolor=blue,citecolor=blue,pdfstartview=FitH]{hyperref}

\usepackage{amsmath,amsfonts}
\usepackage{graphicx}

\bibliographystyle{alpha}

\setlength{\oddsidemargin}{0pt}
\setlength{\evensidemargin}{0pt}
\setlength{\textwidth}{6.0in}
\setlength{\topmargin}{0in}
\setlength{\textheight}{8.5in}

\setlength{\parindent}{0in}
\setlength{\parskip}{5pt}


%%%%%%%%% For wordpress conversion

\def\more{}

\usepackage{ulem}
\def\em{\it}
\let\emph=\it


\newcommand{\coursenum}{CS172}
\newcommand{\coursename}{Automata, Computability and Complexity}
\newcommand{\courseprof}{Professor Luca Trevisan}

%       Usage: \ptitle{title}{dateout}
\newcommand{\ptitle}[2]{\noindent\parbox{\textwidth}
{U.C. Berkeley --- \coursenum : \coursename \hfill #1 \newline
\courseprof \hfill #2 \newline
\mbox{}\hrulefill\mbox{}}\vspace*{1ex}\mbox{}\newline
\bigskip
\begin{center}{\Large\bf #1}\end{center}
\bigskip}


%       Usage: \handout{title}{datelec}{dateout}{scribe}
\newcommand{\handout}[2]{\thispagestyle{empty}
 \markboth{Notes for Lecture #1}{Notes for Lecture #1}
 \pagestyle{myheadings}\htitle{#1}{#2}}

%       Usage: \pset{title}{dateout}
\newcommand{\pset}[2]{\thispagestyle{empty}
 \markboth{#1 --- #2}{#1 --- #2}
 \pagestyle{myheadings}\ptitle{#1}{#2}}


\begin{document}

\classnotes{0}{January 35, 2009}

Edit the {\tt $\backslash$classnotes} command, and enter the correct
lecture number and date.

To the extent possible, do not add new macros. Do not use
{\tt $\backslash$align} and {\tt $\backslash$eqnarray}. If possible,
avoid using the {\tt tabular} environment.

Use {\tt $\backslash$P} as a symbol for probability,
{\tt $\backslash$E} as a symbol for expectation,
{\tt $\backslash$var} for variance,
{\tt $\backslash$Z} for the set of integers, 
{\tt $\backslash$N} for the naturals,
and {\tt $\backslash$R} for the reals.

If you want to include a picture, make sure it is in gif, jpeg, png
or pdf format, and use the {\tt $\backslash$includegraphics$\{$filename$\}$}
command.

\includegraphics[width=4in]{fig1.pdf}

If you used a vector-graphics editor to create the image, send me also
the svg file (or whatever is the standard format for your program).

Use the predefined environments for theorems, lemmas, proofs, remarks, examples, 
and so on.

\end{document}