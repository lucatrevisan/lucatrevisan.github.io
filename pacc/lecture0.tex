\documentclass[11pt]{article}

\usepackage{amssymb,amsmath}

\newcommand{\coursenum}{CS172}
\newcommand{\coursename}{Automata, Computability and Complexity}
\newcommand{\courseprof}{Professor Luca Trevisan}

%       Usage: \ptitle{title}{dateout}
\newcommand{\ptitle}[2]{\noindent\parbox{\textwidth}
{U.C. Berkeley --- \coursenum : \coursename \hfill #1 \newline
\courseprof \hfill #2 \newline
\mbox{}\hrulefill\mbox{}}\vspace*{1ex}\mbox{}\newline
\bigskip
\begin{center}{\Large\bf #1}\end{center}
\bigskip}


%       Usage: \handout{title}{datelec}{dateout}{scribe}
\newcommand{\handout}[2]{\thispagestyle{empty}
 \markboth{Notes for Lecture #1}{Notes for Lecture #1}
 \pagestyle{myheadings}\htitle{#1}{#2}}

%       Usage: \pset{title}{dateout}
\newcommand{\pset}[2]{\thispagestyle{empty}
 \markboth{#1 --- #2}{#1 --- #2}
 \pagestyle{myheadings}\ptitle{#1}{#2}}


\begin{document}

\handout{0}{August 0, 2005}{Scribe: Name of the Scribe}

Some examples of macros

\begin{theorem}[Joe's Theorem] Statement of Joe's Theorem.
\end{theorem}


\begin{proof}
Proof of Joe's theorem.
\end{proof}


\begin{lemma}[Joe's Lemma] Statement of Joe's Lemma.
\end{lemma}

\begin{corollary}[Joe's Corollary] Statement of Joe's Corollary.
\end{corollary}

\begin{definition} [Pseudorandomness]
Let $\cal F$ be a set of functions $f: \B^n \to \B$. We say that
a random variable $X= (X_1,\ldots,X_n)$ distributed over $\B^n$
is $\epsilon$-{\em pseudorandom} for $\cal F$ if 

\begin{equation} 
\forall f \in \F. \hspace{50pt}  \left| \pr_{r\sim \B^n} [f(r)=1] - \pr [ f(X_1,\ldots,X_n) = 1] \right| \leq \epsilon
\end{equation}
\end{definition}

There are also macros for {\em fact}, {\em proposition}, {\em example}, {\em remark}, etc.

To write expectations: $\E_{x \sim \B^n} [ f(x)]$. Note how it looks in equations:

\begin{equation}
\E_{x \sim \B^n} [ f(x)]
\end{equation}

Similarly to write probability: $\pr_{x\sim \B^n} [ f(x) = g(x) ]$.

There are macros for \N, \Z, \R, \B, \H, \F.

If we ever need to write an algorithm, there are some available macros. For example:

\noindent\fbox{\begin{minipage}{2in}
\begin{program}
Graph-Test $(G,f)$\+\\
 Choose uniformly at random $x_1,\ldots,x_{k} \in \H^n$\\
 \IF $f(x_i)f(x_j)f(x_ix_j)=1$ for all $(i,j) \in E$\+\\
 \THEN \ACCEPT\\
 \ELSE \REJECT\-\-
\end{program}\end{minipage}}

% The command \+ increase indentation in the following lines, while \- decreases it.
% Macros for several typical keywords are available, see the code of macros.tex

\bigskip

See also comments in the latex code.

But a nested sequence of {\em itemize} and {\em enumerate}, as appropriate, will also be fine.


\end{document}