\documentclass{article}

\usepackage{fullpage}
\usepackage{amsmath}
\usepackage{amsfonts}
\usepackage{mdwlist}

\newcommand{\coursenum}{CS172}
\newcommand{\coursename}{Automata, Computability and Complexity}
\newcommand{\courseprof}{Professor Luca Trevisan}

%       Usage: \ptitle{title}{dateout}
\newcommand{\ptitle}[2]{\noindent\parbox{\textwidth}
{U.C. Berkeley --- \coursenum : \coursename \hfill #1 \newline
\courseprof \hfill #2 \newline
\mbox{}\hrulefill\mbox{}}\vspace*{1ex}\mbox{}\newline
\bigskip
\begin{center}{\Large\bf #1}\end{center}
\bigskip}


%       Usage: \handout{title}{datelec}{dateout}{scribe}
\newcommand{\handout}[2]{\thispagestyle{empty}
 \markboth{Notes for Lecture #1}{Notes for Lecture #1}
 \pagestyle{myheadings}\htitle{#1}{#2}}

%       Usage: \pset{title}{dateout}
\newcommand{\pset}[2]{\thispagestyle{empty}
 \markboth{#1 --- #2}{#1 --- #2}
 \pagestyle{myheadings}\ptitle{#1}{#2}}

\begin{document}






\ptitle{Problem Set 8}{March 19, 2015}

This problem set is due on Friday, April 3rd, by 5pm. Please submit your solution online using bcourses,
as a pdf file. You can type your solution, or handwrite it. If you handwrite it, then either
scan it or take a good resolution picture of each page and then collate the pictures
and export them to a {\em single} pdf file.

\bigskip

\hrule

\subsection*{Problem 1: NFA Minimization (30)}
In the NFA minimization problem, we are given a NFA $N_1$ and want to output a NFA $N_2$
that accepts the same language using as few states as possible.\\

\noindent Prove that if there is a polynomial time for NFA minimization,
then P = NP.

\subsection*{Problem 2: ExactClique is both NP-hard and coNP-hard (30)}

Let $ExactClique = \{ \langle G, k \rangle \mid \text{ the largest clique in } G \text{ is of size exactly } k \}$.\\

\noindent Prove that $ExactClique$ is both NP-hard and coNP-hard.


\subsection*{Problem 3: Circuit Minimization (40)}
\noindent In the Circuit Minimization problem (CircuitMin),
we are given a boolean circuit $C_1$, and want to output a boolean circuit $C_2$
that accepts the same language using as few gates as possible.
In situations where multiple circuits of minimal size exists,
outputting any circuit of minimal size is acceptable.
Assume the only available gates are: {\bf and :: $ \{0,1\} \times \{0,1\} \rightarrow \{0,1\}$, 
or :: $ \{0,1\} \times \{0,1\} \rightarrow \{0,1\}$, not :: $ \{0,1\} \rightarrow \{0,1\}$}.\\

\noindent Prove that P = NP if and only if CircuitMin can be solved in polynomial time.\\



\end{document}
