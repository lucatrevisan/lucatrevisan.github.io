\documentclass[11pt]{article}
\usepackage{fullpage}
\usepackage{amsfonts}
\usepackage{amssymb}
\usepackage{mdwlist}
\usepackage[usenames,dvipsnames]{xcolor}
\usepackage{tikz}
\usepackage{enumerate}
\usepackage{amsmath}


\newcommand{\coursenum}{CS172}
\newcommand{\coursename}{Automata, Computability and Complexity}
\newcommand{\courseprof}{Professor Luca Trevisan}

%       Usage: \ptitle{title}{dateout}
\newcommand{\ptitle}[2]{\noindent\parbox{\textwidth}
{U.C. Berkeley --- \coursenum : \coursename \hfill #1 \newline
\courseprof \hfill #2 \newline
\mbox{}\hrulefill\mbox{}}\vspace*{1ex}\mbox{}\newline
\bigskip
\begin{center}{\Large\bf #1}\end{center}
\bigskip}


%       Usage: \handout{title}{datelec}{dateout}{scribe}
\newcommand{\handout}[2]{\thispagestyle{empty}
 \markboth{Notes for Lecture #1}{Notes for Lecture #1}
 \pagestyle{myheadings}\htitle{#1}{#2}}

%       Usage: \pset{title}{dateout}
\newcommand{\pset}[2]{\thispagestyle{empty}
 \markboth{#1 --- #2}{#1 --- #2}
 \pagestyle{myheadings}\ptitle{#1}{#2}}



\begin{document}
\ptitle{Problem Set 11}{April 23, 2015}
This problem set is due on Friday, April 30, by 5pm. Please submit your solution online using bcourses,
as a pdf file.
You can type your solution, or handwrite it. If you handwrite it, then either
scan it or take a good resolution picture of each page and then collate the pictures
and export them to a {\em single} pdf file.
\bigskip
\hrule


\section*{Problem 1 (33/100)}
In $\mathbb{Z}_{77}$ find all of the elements that are quadratic residues with only two roots (as opposed to four).  Showing work/reasoning could be helpful in grading.

\noindent Why do these end up not being a problem in the quadratic residuosity scheme we've seen in class?

\noindent [Hint: there are eight such elements]

\section*{Problem 2 (33/100)}
Recall that we defined IP as the class of languages such that for each language L there exists a pair of algorithms (or better, interacting machines) $(\mathcal{P},\mathcal{V})$, where the verifier $\mathcal{V}$ is polynomial in $|x|$ such that:
		\begin{itemize}
			\item Completeness: $\forall x\in L$ 
		$$\Pr\left[Output_{\mathcal{V}}(\mathcal{P}(x) \leftrightarrow \mathcal{V}(x))=1\right]=1$$
			\item Soundness: $\forall x\notin L$, $\forall \mathcal{P}^*$ 
		$$\Pr\left[Output_{\mathcal{V}}(\mathcal{P}^*(x) \leftrightarrow \mathcal{V}(x))=1\right]\leq 1/2$$
		\end{itemize}
		
\begin{enumerate}[a)]
\item Let IP$'$ be the class of languages where we allow the prover to be probabilistic i.e. the prover can use randomness. Show that IP$'$ = IP.

\item Let IP$'$ be the class of languages where we replace the 1/2 in the definition above by 0 i.e. the verifier must surely reject in case $x\notin L$. Show that IP$'$ = NP.
\end{enumerate}

\section*{Problem 3 (34/100)}
Consider the following protocol for showing that $x\in \mathbb{Z}_N$, for $N=pq$, is a quadratic nonresidue i.e. $\nexists y$ such that $x=y^2$ (mod $N$).

\begin{center}
			\includegraphics[scale=.55]{QNR_ZK.png}
\end{center}

That is, 

Verifier: pick random $s\in\mathbb{Z}_N$, then with prob 1/2 send $s^2$ and with prob 1/2 send $s^2 x$. 

Prover: tell whether received number is quadratic residue or not. Note that $QNR(N)$ is the 

\quad\quad\quad\ \ set of quadratic nonresidues mod $N$. 

Verifier: accept if sent $s^2$ and prover says residue or sent $s^2x$ and prover says non-residue.

\noindent Note: for a prover with the factorization of $N$ as its witness, $w=(p,q)$, it is easy for them to determine if $x\in \mathbb{Z}_N$ is a quadratic residue or not.
\\

\noindent Show that the scheme is complete, sound, and honest verifier perfect zero-knowledge, but, unless the quadratic residuosity problem is in polynomial time, the scheme is not perfect zero knowledge.



\end{document}