\documentclass[11pt]{article}
\usepackage{fullpage}
\usepackage{amsfonts}
\usepackage{amssymb}
\usepackage{mdwlist}
\usepackage[usenames,dvipsnames]{xcolor}
\usepackage{tikz}
\usepackage{enumerate}


\newcommand{\coursenum}{CS172}
\newcommand{\coursename}{Automata, Computability and Complexity}
\newcommand{\courseprof}{Professor Luca Trevisan}

%       Usage: \ptitle{title}{dateout}
\newcommand{\ptitle}[2]{\noindent\parbox{\textwidth}
{U.C. Berkeley --- \coursenum : \coursename \hfill #1 \newline
\courseprof \hfill #2 \newline
\mbox{}\hrulefill\mbox{}}\vspace*{1ex}\mbox{}\newline
\bigskip
\begin{center}{\Large\bf #1}\end{center}
\bigskip}


%       Usage: \handout{title}{datelec}{dateout}{scribe}
\newcommand{\handout}[2]{\thispagestyle{empty}
 \markboth{Notes for Lecture #1}{Notes for Lecture #1}
 \pagestyle{myheadings}\htitle{#1}{#2}}

%       Usage: \pset{title}{dateout}
\newcommand{\pset}[2]{\thispagestyle{empty}
 \markboth{#1 --- #2}{#1 --- #2}
 \pagestyle{myheadings}\ptitle{#1}{#2}}



\begin{document}
\ptitle{Problem Set 4}{February 12, 2015}
This problem set is due on Friday, February 20, by 5pm. Please submit your solution online using bcourses,
as a pdf file.
You can type your solution, or handwrite it. If you handwrite it, then either
scan it or take a good resolution picture of each page and then collate the pictures
and export them to a {\em single} pdf file.
\bigskip
\hrule



\section*{Problem 1 (35/100)}
For every integer $n$, consider the streaming complexity of the problem of deciding whether a graph on $n$ vertices, given by a stream of edges, is connected. 

That is, for a set $n$ vertices $V$, our alphabet $\Sigma=\{\{v_1,v_2\}| v_1,v_2\in V,$ $v_1\neq v_2\}$ is all possible (undirected) edges between these vertices  and our stream is a sequence of these edges.  If we call the set of each edge in this stream $E$, then $G=(V,E)$ is the undirected graph defined by it.  We want a streaming algorithm that takes the stream and computes whether or not $G$ is connected.

Show that the bits of memory required for a streaming algorithm for this problem is $\Omega(n)$.

\section*{Problem 2 (30/100)}
Create a Turing Machine that can decide the language of palindromes 

$L=\{x\in\Sigma^*=\{a,b\}^*|x= $reverse$(x)\}$ 

\noindent(Note that this was impossible with DFAs)

\section*{Problem 3:  (35/100)}
Consider a read-only Turing Machine; that is, a Turing Machine that cannot write to the tape.  Formally, this means that the transition function's output is no longer a triple since it can no longer write a symbol from $\Gamma$ to the tape and instead only transitions to a different state and moves left or right on the tape.  So, $\delta: Q \times \Gamma \rightarrow Q \times \{L,R\}$ and, moreover, $\Gamma=\Sigma\cup\{\square\}$ since there are no ``special" symbols for the Turing Machine to write.

Equivalently, a read-only Turing Machine may be thought of as a regular Turing Machine whose transition function is defined so that the only thing it can write is what it just read: $\forall\gamma\in\Gamma$ $\delta(q,\gamma)=(q',\gamma,direction)$.  In this way writing is superfluous and all of its power is from reading the tape and being able to move both left and right.

Show that, despite the ability to move backwards on the tape, a read-only Turing Machine (according to either definition) can only decide regular languages.

\end{document}