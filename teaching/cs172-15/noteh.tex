\documentclass[11pt]{article}
\usepackage{amsfonts}
\usepackage{amssymb}

%\usepackage{doublespace}
%\setstretch{1.15}


%Formatting


\setlength{\topmargin}{0in}
\setlength{\oddsidemargin}{0 in}
\setlength{\evensidemargin}{0 in}
\setlength{\textwidth}{6.5 in}
\setlength{\textheight}{8.5 in}
\setlength{\headsep}{0.75 in}
\setlength{\parindent}{0 in}
\setlength{\parskip}{0.05 in}

%Theorems, proofs, etc.

\newtheorem{theorem}{Theorem}
\newtheorem{lemma}[theorem]{Lemma}
\newtheorem{corollary}[theorem]{Corollary}
\newtheorem{proposition}[theorem]{Proposition}
\newtheorem{claim}[theorem]{Claim}
\newtheorem{exercise}{Exercise}
\newtheorem{conjecture}{Conjecture}
\newtheorem{example}{Example}
\newtheorem{remark}{Remark}
\newtheorem{definition}[theorem]{Definition}
\newenvironment{proof}{\noindent {\sc Proof:}}{$\Box$\medskip} 
\newtheorem{question}{Open Question}
\newtheorem{problem}{Problem}



%% Macros for complexity theory notation
%% Complexity classes
\newcommand\p{\mbox{\bf P}}
\newcommand\np{\mbox{\bf NP}}
\newcommand\am{\mbox{\bf AM}}
\newcommand\ma{\mbox{\bf MA}}
\newcommand\distnp{\mbox{\bf DistNP}}
\newcommand\cnp{\mbox{\bf coNP}}
\newcommand\corp{\mbox{\bf coRP}}
\newcommand\sigmatwo{\mbox{\bf $\Sigma_2$}}
\newcommand\sigmathree{\mbox{\bf $\Sigma_3$}}
\newcommand\pitwo{\mbox{\bf $\Pi_2$}}
\newcommand\rp{\mbox{\bf RP}}
\newcommand\zpp{\mbox{\bf ZPP}}
\newcommand\bpp{\mbox{\bf BPP}}
\newcommand\ph{\mbox{\bf PH}}
\newcommand\pspace{\mbox{\bf PSPACE}}
\newcommand\npspace{\mbox{\bf NPSPACE}}
\newcommand\aczero{\mbox{\bf AC$^0$}}
\newcommand\dl{\mbox{\bf L}}
\newcommand\nl{\mbox{\bf NL}}
\newcommand\rl{\mbox{\bf RL}}
\newcommand\sll{\mbox{\bf SL}}
\newcommand\bpl{\mbox{\bf BPL}}
\newcommand\conl{\mbox{\bf coNL}}
\newcommand\sharpp{\mbox{\#{\bf P}}}
\newcommand\parityp{\mbox{$\oplus$ {\bf P}}}
\newcommand\ip{\mbox{\bf IP}}
\newcommand\pcp{\mbox{\bf PCP}}
\newcommand\dtime{\mbox{\bf DTIME}}
\newcommand\ntime{\mbox{\bf NTIME}}
\newcommand\bptime{\mbox{\bf BPTIME}}
\newcommand\dspace{\mbox{\bf SPACE}}
\newcommand\nspace{\mbox{\bf NSPACE}}
\newcommand\cnspace{\mbox{\bf coNSPACE}}
\newcommand\exptime{\mbox{\bf EXPTIME}}
\def\exp{\mbox{\bf EXP}}
\def\ep{\mbox{\bf E}}
\newcommand\nexptime{\mbox{\bf NEXPTIME}}
\newcommand\genclass{\mbox{$\cal C$}}
\newcommand\cogenclass{\mbox{\bf co$\cal C$}}
\newcommand\size{\mbox{\bf SIZE}}

\def\phi{\varphi}

%%Computational problems

\newcommand\sat{\mbox{SAT}}
\newcommand\tsat{\mbox{3SAT}}
\newcommand\tqbf{\mbox{TQBF}}
\newcommand\permp{\mbox{\sc Permanent}}
\def\perm{\operatorname{perm}}

%% Notation for integers, natural numbers, reals, fractions, sets, cardinalities
%%and so on

\newcommand\Z{\mbox{$\mathbb Z$}}
\newcommand\N{\mbox{$\mathbb N$}}
\newcommand\R{\mbox{$\mathbb R$}}
\newcommand\F{\mbox{$\mathcal{F}$}}
\newcommand\B{\{0,1\}}      % boolean alphabet  use in math mode
\newcommand\Bs{\{0,1\}^*}   % B star            use in math mode
\def\H{\{-1,1\}}      
\def\S{{\cal S}}

\newcommand{\bzero}{{\bf 0}}


\newcommand\true{\mbox{\sc True}}
\newcommand\false{\mbox{\sc False}}

\newcommand{\marginlabel}[1]%
{\mbox{}\marginpar{\it{\raggedleft\hspace{0pt}#1}}}
\newcommand\poly{\mbox{poly}}  %usage \poly(n)
\newcommand{\floor}[1]{\lfloor\, $#1$\,\rfloor}
\newcommand{\ceil}[1]{\lceil\, $#1$\,\rceil}
\newcommand{\comp}[1]{\overline{#1}}


%\newcommand\to{\rightarrow} 
\newcommand\xor{\oplus}
\newcommand\bigxor{\bigoplus}
\newcommand{\logred}{\leq_{\log}}
\def\implies{\Rightarrow}


%% probability stuff

\newcommand\E{\mathop\mathbb{E}}
\newcommand\pr{\mathop{\bf {Pr}}}
\newcommand\var{\mathop{\bf Var}}


%% macros to write pseudo-code

\newlength{\pgmtab}  %  \pgmtab is the width of each tab in the
\setlength{\pgmtab}{1em}  %  program environment
 \newenvironment{program}{\renewcommand{\baselinestretch}{1}%
\begin{tabbing}\hspace{0em}\=\hspace{0em}\=%
\hspace{\pgmtab}\=\hspace{\pgmtab}\=\hspace{\pgmtab}\=\hspace{\pgmtab}\=%
\hspace{\pgmtab}\=\hspace{\pgmtab}\=\hspace{\pgmtab}\=\hspace{\pgmtab}\=%
\+\+\kill}{\end{tabbing}\renewcommand{\baselinestretch}{\intl}}
\newcommand {\BEGIN}{{\bf begin\ }}
\newcommand {\ELSE}{{\bf else\ }}
\newcommand {\IF}{{\bf if\ }}
\newcommand {\FOR}{{\bf for\ }}
\newcommand {\TO}{{\bf to\ }}
\newcommand {\DO}{{\bf do\ }}
\newcommand {\WHILE}{{\bf while\ }}
\newcommand {\ACCEPT}{{\bf accept}}
\newcommand {\REJECT}{\mbox{\bf reject}}
\newcommand {\THEN}{\mbox{\bf then\ }}
\newcommand {\END}{{\bf end}}
\newcommand {\RETURN}{\mbox{\bf return\ }}
\newcommand {\HALT}{\mbox{\bf halt}}
\newcommand {\REPEAT}{\mbox{\bf repeat\ }}
\newcommand {\UNTIL}{\mbox{\bf until\ }}
\newcommand {\TRUE}{\mbox{\bf true\ }}
\newcommand {\FALSE}{\mbox{\bf false\ }}
\newcommand {\FORALL}{\mbox{\bf for all\ }}
\newcommand {\DOWNTO}{\mbox{\bf down to\ }}


%%%%%%%%    FALL'05
\newcommand{\coursenum}{CS294}
\newcommand{\coursename}{Pseudorandomness and Combinatorial Constructions}
\newcommand{\courseprof}{Professor Luca Trevisan}



%%%%%%%%%%%%%%%%%%%%%%%%%%%%%%%%%%%%%%%%%%%%%%%%%%%%%%%%%%%%%%%%%%%%%%%%%%%
%%%%%%%%%%%%%%%%%%%%%%%%%%%%%%%%%%%%%%%%%%%%%%%%%%%%%%%%%%%%%%%%%%%%%%%%%%%




\newlength{\tpush}
\setlength{\tpush}{2\headheight}
\addtolength{\tpush}{\headsep}


\newcommand{\handout}[3]{\noindent\vspace*{-\tpush}\newline\parbox{\textwidth}
{U.C. Berkeley  \hfill Handout N#1 \newline
\coursenum : \coursename \hfill #2 \newline
\courseprof \hfill #3 \newline
\mbox{}\hrulefill\mbox{}}\vspace*{1ex}\mbox{}\newline
\bigskip
\begin{center}{\Large\bf Notes for Lecture #1}\end{center}
\bigskip}


\newcommand{\problemset}[2]{\noindent\vspace*{-\tpush}\newline\parbox{\textwidth}
{U.C. Berkeley --- \coursenum : \coursename \hfill Handout #1 \newline
\courseprof \hfill #2 \newline
\mbox{}\hrulefill\mbox{}}\vspace*{1ex}\mbox{}\newline
\bigskip
\bigskip}

















\begin{document}

\sloppy

\htitle{8}{Notes on Hierarchy Theorems}{4/21/2015}



\bigskip

\noindent
These notes discuss the proofs of the time and space hierarchy theorems.
It shows why one has to work with the rather counter-intuitive diagonal
problem defined in Sipser's book, and it describes an alternative
proof of the time hierarchy theorem.


When we say ``Turing machine'' we mean one-tape Turing machine
if we discuss time. If we discuss 
space, then we mean a Turing machine with one read-only tape
and one work tape.
All the languages discussed in these notes are over the alphabet
$\Sigma = \B$.

\begin{definition}[Constructibility]
A function $t:\nat \to \nat$ is {\em time constructible}
if there is a Turing machine $M$ that given an input of
length $n$ runs in $O(t(n))$ time and writes $t(n)$ in
binary on the tape.
A function $s:\nat\to\nat$ is {\em space constructible}
if there is a Turing machine $M$ that given an input
of length $n$ uses $O(s(n))$ space and writes $s(n)$
on the (work) tape.
\end{definition}


\begin{theorem}[Space Hierarchy Theorem]
If $s(n)\geq \log n$ is a space-constructible function then
$\dspace(o(s(n)) \neq \dspace (O(s(n))$.
\end{theorem}



\begin{theorem}[Time Hierarchy Theorem]
If $t(n)$ is a time-constructible function such that $t(n)\geq n$ and
 $n^2 = o(t(n))$,  then
$\dtime (o(t(n)) \neq \dtime (O(t(n) \log t(n)))$.
\end{theorem}

\section{Proof of The Space Hierarchy Theorem}

Let $s:\nat \to \nat$ be a space constructible function such that $s(n) \geq \log n$.

We want to define a language $L$ that can be solved
in space $O(s(n))$ but not in space $o(s(n))$. The basic idea
would be to define the language

\[ L_1 = \{ ( \langle M \rangle, w) : \mbox{ $M$ rejects 
$( \langle M \rangle, w) $ using $\leq s(|\langle M \rangle, w|)$
space } \} \]
Then no machine $M'$ can solve $L_1$ using less than $s(n)$ space, because
otherwise we would get a contradiction on all computations of $M'$ on input
$( \langle M' \rangle, w)$. On the other hand, deciding $L_1$ amounts to 
simulating a machine that uses $\leq s(n)$ space and it looks like
such a simulation can be performed using $O(s(n))$ space.

Unfortunately, there are some difficulties in devising an $O(s(n))$ space
algorithm for $L_1$. 

One  problem is that the machine $M$ is part of the input,
and so on an input of length $n$ the description of $M$ could be, say, $n/2$ bits
long. The description of $M$ includes a description of $M$'s tape alphabet, and
so the tape alphabet of $M$ might have a large (depending on $n$) number of
symbols, like, say, $\sqrt n$. In a computation of $M$ in which $s(n)$ entries
of tape are used, the tape is actually holding $O(s(n) \cdot \log n)$ bits of
information, and if we want to simulate such a computation using a fixed
Turing machine $M_{L_1}$ with a finite alphabet it seems unavoidable that $M_{L_1}$
will use more than $O(s(n))$ cells of tape.

Another problem is that we want to {\em decide} the language $L_1$, that is, 
we want to design a machine $M_{L_1}$ that uses as little space as possible
and that, on every input $(\langle M \rangle , w)$, halts with the right answer.
But if $M_{L_1}$ just simulates $M$ on input $(\langle M \rangle , w)$, then $M_{L_1}$
will never halt if $M$ loops on input $(\langle M \rangle , w)$.

We solve the first problem by changing slightly the definition of $L_1$. We
solve the second problem by introducing a counter of the number of steps
used in the simulation of $M$, and aborting the simulation if the number
of steps is too large.

To solve the first problem, we recall a result that we discussed several lectures 
ago when talking about Turing machine.

\begin{lemma}[Size of Alphabet Tape] \label{lm:alpha}
Let $L$ be a language decided by a Turing machine $M = (Q,\B, \Gamma,\delta,q_0,q_A,q_R)$. Then there is
another Turing machine $M'=(Q',\B, \Gamma',\delta',q'_0,q'_A,q'_R)$
that also decides $L$ and whose tape alphabet $\Gamma'$ has only four elements.
Furthermore, if $M$ uses space $s(n)$ then $M'$ uses space $O(s(n))$.
\end{lemma}

The ``furthermore'' part is new, but it follows from the same proof.
Now, instead of $L_1$, consider the following language.

\begin{eqnarray*}
 L_2 &=  \{ ( \langle M \rangle, w) : & \mbox{ $M$ rejects $( \langle M \rangle, w) $
using $\leq s(|\langle M \rangle, w|)$
space} \\ && \mbox{ and  $M$'s tape alphabet has size four } \} \end{eqnarray*}

\begin{lemma}
$L_2 \not\in \dspace(o(s(n))$.
\end{lemma}

\begin{proof}
Suppose there is a Turing machine $M'$ that solves $L_2$
using $o(s(n))$ space. Then, using Lemma \ref{lm:alpha},
there is a Turing machine $M''$ with a tape alphabet of size four
that solves $L_2$ using $o(s(n))$ space. For sufficiently long
$w$,  $M''$ uses less than $s(|\langle M'' \rangle, w|)$ space
on input $\langle M'' \rangle, w$, and now we get a contradiction
no matter whether we assume $(\langle M'' \rangle, w) \in L_2$ or not. 
\end{proof}

\begin{lemma}
$L_2 \in \dspace(O(s(n) + \log n))= \dspace(O(s(n))$.
\end{lemma}

\begin{proof}
We are given in input a machine $M= (Q,\B,\Gamma,\delta,q_0,q_A,q_R)$ with
$|\Gamma | = 4$ and a string $w$, we let $n$ be the total length of 
$\langle M \rangle ,w$. Note that $|Q| \leq n$.

We first compute $s(n)$ and mark $s(n)$ cells of work tape. This can be
done using $O(s(n))$ space because of the space-constructibility of $s()$.
Then we compute the number $t = |Q|\cdot 4^{s(n)} \cdot n$, and we initialize a
counter to the value $t$, written in binary on the work tape.
The counter uses space that is separate from the $s(n)$ cells that we have marked before.
Writing the number $t$ requires $O(s(n) + \log n)$ bits, and $t$ can be
computed using space $O(s(n) + \log n)$. Also, $t$ is an upper bound
to the number of configurations of machine $M$ given an input of length $n$,
assuming that $M$ uses $\leq s(n)$ space on that input.

Then we proceed to a simulation of the computation of  $M$
on input   $\langle M \rangle ,w$. We use the $s(n)$ marked
position of work tape to store the work  tape of $M$ and another $\log |Q|
= O(\log n)$ positions of work tape to store the current state of $M$.
After each step of the simulation, we decrease the counter.

\begin{itemize}

\item If the counter ever reaches zero, then
$M$ has gone  through more than $t$ steps on an input of length $n$ and
using space $s(n)$, which means that it must have gone through the same configuration
twice, and so it loops on that input. If $M$ loops on input
$\langle M \rangle ,w$ then $(\langle M \rangle ,w)\not\in L_2$.
Therefore, if the counter reaches zero, we reject.

\item If the $s(n)$ cells allocated for the simulation of the tape 
of $M$ are not sufficient, that is, if $M$ uses more than $s(n)$
space on input $\langle M \rangle ,w$ then $(\langle M \rangle ,w)\not\in L_2$,
and so we reject.

\item If the simulation ends within $t$ steps, then if $M$ accepts we reject,
and if $M$ rejects we accept.

\end{itemize}

Note that, in all possible cases, our algorithm terminates, decides $L_2$ correctly,
and uses $O(s(n) + \log n) = O(s(n))$ space.
\end{proof}


\section{Two Proofs of the Time Hierarchy Theorem}

Let $t:\nat \to \nat$ be a time constructible function such that $t(n)\geq n$
and $n^2 = o(t(n)$.

We want to define a language $L$ that can be solved
in time  $O(t(n)\log t(n))$ but not  in time $o(t(n))$. Continuing
with the intuition from last section, the first idea would be to
define a language

\[ L_3 = \{ ( \langle M \rangle, w) : \mbox{ $M$ rejects $w$ in $\leq t(|\langle M \rangle, w|)$
time steps } \} \]
Then no machine $M'$ can solve $L_3$ using less than $t(n)$ time, because
otherwise we would get a contradiction on all computations of $M'$ on input
$( \langle M' \rangle, w)$. As before, the difficult part is to give
a good algorithm for $L_3$. Doing a time-efficient simulation is even
harder than doing a space-efficient simulation, because one-tape Turing
machine are very hard to program in a time-efficient way. If the size
of $\langle M \rangle$ is, say, $n/2$, it may take time more than $O(n)$ to do
even one step of a simulation of $M$ on a given input. However, at least
the following result is easy to prove.

\begin{lemma}\label{lm:univ}
There is a universal Turing machine $U$ that, on input $( \langle M \rangle, w)$,
where $M$ is a Turing machine with tape alphabet of size four,
simulates the computation of $M ( \langle M \rangle, w)$. Every step of the
simulation takes time $O(|\langle M \rangle| ^2)$.

In other words, if $M ( \langle M \rangle, w)$ loops then $U ( \langle M \rangle, w)$
loops. If $M ( \langle M \rangle, w)$ accepts (respectively, rejects)
in $t$ steps then $U ( \langle M \rangle, w)$ accepts (respectively, rejects)
in $O(|\langle M \rangle |^2 \cdot t)$ steps.
\end{lemma}

\begin{proof}
We organize the tape into two tracks (see the proof theorem 9.10 in 
Sipser) and we put a copy of $\langle M \rangle$ and the initial state
$q_0$ of $M$ on the second track. The first track contains the initial input.
At every step, we see what is the symbol being read on the first track,
what is the state, and we search $\langle M\rangle$ for a description of
the next state, the next symbol and the move. We update the current state on track two
and the element on track one accordingly, we move the state and $\langle M \rangle$ one
position left of right as required, and we proceed with the next step of the
simulation  
\end{proof}

Regarding the universality of machines with small tape alphabet we have.

\begin{lemma}
Let $L$ be a language decided in time $t(n)$ by a turing machine $M$.
Then $L$ is decided in time $O(t(n) + n^2)$ by a Turing machine $M'$
whose tape alphabet has size four.
\end{lemma}


We then change the definition of $L_3$ in order to force the description of
the Turing machine $M$ to be short.

\begin{eqnarray*}
 L_4 & =   & \{( \langle M \rangle, w) : \mbox{ $M$ rejects $w$ in $\leq t(|\langle M \rangle, w|)$
time steps},\\ && |\langle M \rangle | \leq \sqrt{\log t(|\langle M \rangle, w|)} \mbox{ and $M$'s 
tape alphabet has size 4} \} \end{eqnarray*}

\begin{lemma}
$L_4 \not\in \dtime (o(t(n)))$.
\end{lemma}

\begin{proof}
Suppose $L_4$ can be solved by a machine $M'$ in $o(t(n))$ time.
Then it can also be solved by a machine $M''$ with a tape alphabet
of size 4 in $o(t(n)) + O(n^2) = o(t(n))$ time. Let $k$
be length of the description of $M''$. Let 
$w$ be a string of length $2^{k^2}$. Then 
\[ |\langle M'' \rangle | \leq \sqrt{\log t(|\langle M'' \rangle, w|)} \]
and we get a contradiction when we reason about whether the
string $(\langle M'' \rangle, w)$ should be in $L_4$ or not. 
\end{proof}

\begin{lemma} \label{lm:simtime}
$L_4 \in \dtime (O(t(n)\cdot \log t(n)))$.
\end{lemma}

\begin{proof}
We are given in input a machine $M$ and a string $w$; the total length of
the input is $n$. If the length of the description of $M$ is more than $\sqrt{\log t(n)}$,
we reject. Otherwise, we proceed as follows.

We first organize the tape into three tracks, as described in the proof of 
Theorem 9.10 in Sipser. Two tracks are as in the proof of Lemma \ref{lm:univ},
and the third track stores a counter initialized at $t(n)$. After every step of
the simulation, we decrease the counter, and then we move the counter so that
it is always on track with the copy of $\langle M \rangle$ and with the state
on track two. Updating and moving the counter adds an overhead of $O(\log t(n))$ time
to each step of the simulation, in addition to the $O(|\langle M \rangle |^2 = O(\log t(n))$
time that it takes to find and simulate the correct transition. 

If the counter reaches zero we reject; if the simulation ends before the
counter reaches zero, then we accept if $M$ rejects and we reject if $M$
accepts. We need to simulate at most $t(n)$ steps, and so 
the total running time is $O(t(n) \log t(n))$.
\end{proof}


The proof of the time hierarchy theorem in Sipser's book is slightly different.
Sipser defines the language

\[
 L_5 =    \{( \langle M \rangle, w) : \mbox{ $U$ rejects $(\langle M \rangle ,w)$ in 
$\leq t(|\langle M \rangle, w|)$
time steps} \} \]
\noindent Where $U$ is the machine of Lemma \ref{lm:univ}.
Then we can make the following claims.

\begin{lemma}
$L_5 \not\in \dtime (o(t(n)))$.
\end{lemma}

\begin{proof}
Suppose some machine $M'$ solves $L_5$ in $o(t(n))$ time, then
some machine $M''$ with an alphabet tape of size 4 solves $L_5$
also in $o(t(n))$ time, and $U$ runs in $o(t(n))$ time
given inputs of the form $\langle M'' \rangle , w$ of length $n$.

Let $w$ be large enough so that $U$ runs in time $\leq t(n)$
given in input $(\langle M'' \rangle , w)$ of length $n$. 
Then we reach a contradiction whether or not we assume that
this input is in $L_5$ or not. 
\end{proof}

\begin{lemma}
$L_5 \in \dtime (O(t(n)\log t(n)))$.
\end{lemma}

\begin{proof}
This is similar to the proof of Lemma \ref{lm:simtime}, except that it is
simpler because we only need to simulate a particular, fixed Turing machine.
We can use two tracks, one being the tape of $U$ and the other storing
a counter initialized to $t(n)$. The counter is decremented and moved
every time a step of $U$ is simulated. If the counter reaches zero we reject.
If the simulation ends within $t(n)$ simulated steps then we reject if $U$
accepts and we accept if $U$ rejects. Each of the $\leq t(n)$ steps of the
simulation takes time $O(\log t(n))$, and so the total running time
is $(O(t(n)\log t(n)))$.
\end{proof}


\end{document}
