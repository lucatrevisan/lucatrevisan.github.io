\documentclass[11pt]{article}
\usepackage
[pdftex,pagebackref,letterpaper=true,colorlinks=true,pdfpagemode=none,urlcolor=blue,linkcolor=blue,citecolor=blue,pdfstartview=FitH]{hyperref}

\usepackage{amsmath,amsfonts}
\usepackage{graphicx}

\bibliographystyle{alpha}

\setlength{\oddsidemargin}{0pt}
\setlength{\evensidemargin}{0pt}
\setlength{\textwidth}{6.0in}
\setlength{\topmargin}{0in}
\setlength{\textheight}{8.5in}

\setlength{\parindent}{0in}
\setlength{\parskip}{5pt}


%%%%%%%%% For wordpress conversion

\def\more{}

\usepackage{ulem}
\def\em{\it}
\let\emph=\it


\newcommand{\coursenum}{CS172}
\newcommand{\coursename}{Automata, Computability and Complexity}
\newcommand{\courseprof}{Professor Luca Trevisan}

%       Usage: \ptitle{title}{dateout}
\newcommand{\ptitle}[2]{\noindent\parbox{\textwidth}
{U.C. Berkeley --- \coursenum : \coursename \hfill #1 \newline
\courseprof \hfill #2 \newline
\mbox{}\hrulefill\mbox{}}\vspace*{1ex}\mbox{}\newline
\bigskip
\begin{center}{\Large\bf #1}\end{center}
\bigskip}


%       Usage: \handout{title}{datelec}{dateout}{scribe}
\newcommand{\handout}[2]{\thispagestyle{empty}
 \markboth{Notes for Lecture #1}{Notes for Lecture #1}
 \pagestyle{myheadings}\htitle{#1}{#2}}

%       Usage: \pset{title}{dateout}
\newcommand{\pset}[2]{\thispagestyle{empty}
 \markboth{#1 --- #2}{#1 --- #2}
 \pagestyle{myheadings}\ptitle{#1}{#2}}

\begin{document}
\handout{Manual for scribes}{}{}{}

You will need to use the latex document processing system.
This document is not a latex manual, but it does show you some basic tricks.

A file called {\tt template.tex}
 already exists in this directory that wil take care
of formatting. A file called {\tt macros.tex} contains shorthands for 
complexity classes, frequently used mathematical formatting tricks and the like.
Double-check with me before editing either of these. Also, send me email if you
have latex questions. 

We are using the {\tt amsmath} package for typesetting, which does a few
things differently from standard latex.

Some tricks and tips appear below.
Look in the source file {\tt howto.tex} to understand how
to do these tricks.

\section{General Formatting Tips}

\begin{enumerate}

\item Theorems, lemmas, corollaries, proofs, definitions, examples, exercises,
remarks, etc. are typeset
inside special environments. (The environment names are 
{\tt Thm, Lem, Cor, proof, Def, Exa, Ex, Rem} respectively.) 
Here is how you write a theorem.

\begin{Thm} \label{einsteinthm}
If $E$ denote energy, $m$ denotes mass, and $c$ denotes the speed of light,
then
\begin{equation}
E = mc^2
\end{equation}
\end{Thm}

\item The file {\tt macros.tex} contains macros for names of 
most complexity classes such as $\p, \np, \sharpp, \pspace$. 
If you don't see a macro for some complexity class that
was covered in the lecture, {\em please send me email}. Do not edit 
{\tt macros.tex} or add a macro of your own.

\item File {\tt macros.tex} also contains macros to typeset the following
(not an exhaustive list): set notation (e.g. $\set{1, 2, 3, 4}$),
cardinality of a set (e.g. $\card{\set{1,2,3}}$), Real and natural numbers
($\rea, \nat$ respectively), probabilities (e.g. $\pr[\text{coin comes up head}] =1/2$), $\var[X] = \av[X^2] - \av[X]^2$). 

\item There are macros for writing pseudocode. Look in the
source file to see how to generate the following pice of
pseudocode.

\begin{program}
input: $G = (V,E)$, $s$, $t$ \\
output: \textsc{yes} if it discovers that $t$ is  reachable from $s$, 
and \textsc{no} otherwise \\
\\
        \>     guess the distance $k$ between $s$ and $t$\\
        \>     $p$ := $s$\\
    	\>     \FOR \= $i$ := 1 to $k$ \DO \\
        \>     \> non-deterministically pick a neighbor $q$ of $p$ \\
        \>     \> $p$ := $q$ \\
        \>     \IF\ $p=t$ \THEN\ \ACCEPT\\ 
        \>     \> \ELSE \REJECT 
\\
\end{program}


\item You can include figures by using the {\tt ffigure} command.
You first create a figure using {\tt xfig} (on Unix)
or {\tt Adobe Illustrator},
save it as a postscript file (the subscript should be {\tt .eps} in the same
directory as the latex files. Lets say this figure is {\tt lecture6fig3.eps}. 
Look
in the source file to see how we can include this file and generate
Figure \ref{figure:function}.

\ffigure{lecture6fig3}{3in}
{A circuit computing any function $f(x_1 x_2 \ldots x_n)$ of $n$ variables assuming circuits for two functions of $n-1$ variables.}
{figure:function}




If you get any error messages while including figures, check that the .eps file
begins with \%! and if not, edit out the first line or two of junk.

You will probably get better results if you draw the figures (in {\tt xfig} or
another program) in landscape orientation. Make it fill the entire page,
since you can resize it when using the {\tt ffigure} command.

\end{enumerate}

\section{General Math Formatting Tips}

\begin{enumerate}
\item Use {\tt align} to typeset a series of contiguous equations.
(Do not use the old {\tt eqnarray} command; it uses nonstandard 
typographical conventions.) In the source file you will see that 
an \& tells the program which symbol to align on.
\begin{align}
E & =  mc^2\\
E+ H + G & = t 
\end{align}


Use the  {\tt equation} command for single equations.
\begin{equation}
E  =  mc^2
\end{equation}

To mix text into math formulae, use the {\tt text} command.
\begin{equation}
E  =  mc^2 \qquad \text{(Einstein)}
\end{equation}

While  presenting a sequence of calculations 
(using the {\tt align} command) we sometimes
need to say something briefly in the middle, say to explain a step.
We can do this with
the {\tt intertext} command.
\begin{align}
A+ B+ C + D + E & =  R+ S \\
\intertext{{\em intertext:} which can be upperbounded using the inductive hypothesis by}
& \leq Q + N
\end{align}

\item If no alignment is needed, we use {\tt gather} to make the group of equations look
neat.

\begin{gather}
a+b = b+a \\
(a+b)\cdot (a-b) = a^2 -b^2
\end{gather}

\item There is also {\tt alignat} for {\tt align} type structures side by side.

\begin{alignat}{2}
L_1 & = R_1 & \qquad L_2 & = R_2 \\
L_3 & =  R_3 & \qquad L_4 &= R_4
\end{alignat}

\item Equations that do not fit into a line are typeset using the {\tt split} environment,
which allows alignment between lines using \& as usual.
\begin{equation}
\begin{split}
(a+ b)^3 - (c+d)^3 - (a+ d)^3  & = a^3 + b^3 +3ab(a+b) + c^3 +d^3 +3cd(c+d) \\
& \quad - (a^3 + d^3 +3ad(a+d))
\end{split}
\end{equation}
\item To refer later to an equation, you need to label it with a {\tt label} command.
The command {\tt notag} will make the equation unnumbered. The command {\tt tag} will
replace the equation number with some other designated symbol.

\begin{align}
x^2 -y^2 &= (x-y)\cdot(x+y) \label{eq:r1} \\
x^3 -y^3 & = (x-y)(x^2 +xy + y^2). \tag{$*$} \label{mystar}\\
\intertext{Using \eqref{eq:r1} and \eqref{mystar} we obtain}
a + b &= d  \\
\intertext{Now we give an unnumbered equation; note that the numbering resumes below}
d+ e & = f \notag
\end{align}
\item You can typeset equations involving ``case'' situations with the {\tt cases}
environment.
\begin{equation}
\delta_{i,j} = \begin{cases}
	1 & \text{if $i =j$} \\
	-1 & \text{if $i<j$} \\
	0 & \text{otherwise}
	\end{cases}
\end{equation}
\item The {\tt matrix} environment produces matrices. Below, we show the 
matrices produced using {\tt matrix, pmatrix, bmatrix, vmatrix, Vmatrix} respectively.
\begin{gather*}
\begin{matrix} a & b \\ c & d \end{matrix} \quad
\begin{pmatrix} a & b \\ c & d \end{pmatrix} \quad 
\begin{bmatrix} a & b \\ c & d \end{bmatrix} \quad
\begin{vmatrix} a & b \\ c & d \end{vmatrix} \quad
\begin{Vmatrix} a & b \\ c & d \end{Vmatrix}
\end{gather*}
This environment can handle matrices with up to $10$ columns.
To produce a matrix like 
$\left( \begin{smallmatrix} a & b \\ c& d \end{smallmatrix} \right)$ 
inside a text
paragraph, use the {\tt smallmatrix} environment and enclose with the appropriate
parentheses. Try not to do this in the last line of the paragraph since that looks
untidy, as here $\left( \begin{smallmatrix} a & b \\ c& d \end{smallmatrix} \right)$.

\item To use bold-faced letters inside math equations, use {\tt mathbf} command.
To get bold numbers, greek symbols etc., use the {\tt boldsymbol}
command.
\begin{equation}
\mathbf{A}_{\boldsymbol{\infty}} = \boldsymbol{\alpha}+ d
\end{equation}

\item Function and operator symbols (in addition to standard ones like {\tt sin, cos, log})
can be defined using the {\tt operatorname} command. 

\item Typeset modular arithmetic using the {\tt pmod} command.
\begin{equation}
x \equiv y + z \pmod{n}.
\end{equation}
\item You may be familiar with the {\tt frac} command in latex. In practice one needs
to distinguish typographically between fractions inside paragraph (such as
$\tfrac{x+y}{z^2 +5}$) and  ones that occur in displayed equations, such as
\begin{equation}
\dfrac{x+ y}{z^2 + 5} = 8
\end{equation}
The former use the {\tt tfrac} command and the latter the {\tt dfrac} command.
One can also specify the parentheses around displayed fractions with 
{\tt fracp} and {\tt fracb} commands. The advantage of
these is that the size of the parentheses
will change automatically with the size of the display; compare $\fracp{x+y}{z^2 +5}$with 
$$
\fracp{\fracp{x}{u}+y}{z^2 +5} = 5 = \fracb{x/u+y}{z^2 + 5}
$$
%({\em This is my hack, but apparently there used to be  standard commands like this.})

\item Random examples of how to format some math expressions: $\binom{n}{k}$,
$\exp(-x^2)$, $x \wedge y \vee \overline{z}$, $f \from \nat \to \rea$.

\end{enumerate}
\end{document}









