\documentclass[11pt]{article}
\usepackage
[pdftex,pagebackref,letterpaper=true,colorlinks=true,pdfpagemode=none,urlcolor=blue,linkcolor=blue,citecolor=blue,pdfstartview=FitH]{hyperref}

\usepackage{amsmath,amsfonts}
\usepackage{graphicx}

\bibliographystyle{alpha}

\setlength{\oddsidemargin}{0pt}
\setlength{\evensidemargin}{0pt}
\setlength{\textwidth}{6.0in}
\setlength{\topmargin}{0in}
\setlength{\textheight}{8.5in}

\setlength{\parindent}{0in}
\setlength{\parskip}{5pt}


%%%%%%%%% For wordpress conversion

\def\more{}

\usepackage{ulem}
\def\em{\it}
\let\emph=\it



%Formatting


\setlength{\topmargin}{0in}
\setlength{\oddsidemargin}{0 in}
\setlength{\evensidemargin}{0 in}
\setlength{\textwidth}{6.5 in}
\setlength{\textheight}{8.5 in}
\setlength{\headsep}{0.75 in}
\setlength{\parindent}{0 in}
\setlength{\parskip}{0.05 in}

%Theorems, proofs, etc.

\newtheorem{theorem}{Theorem}
\newtheorem{lemma}[theorem]{Lemma}
\newtheorem{corollary}[theorem]{Corollary}
\newtheorem{proposition}[theorem]{Proposition}
\newtheorem{claim}[theorem]{Claim}
\newtheorem{exercise}{Exercise}
\newtheorem{conjecture}{Conjecture}
\newtheorem{example}{Example}
\newtheorem{remark}{Remark}
\newtheorem{definition}[theorem]{Definition}
\newenvironment{proof}{\noindent {\sc Proof:}}{$\Box$\medskip} 
\newtheorem{question}{Open Question}
\newtheorem{problem}{Problem}



%% Macros for complexity theory notation
%% Complexity classes
\newcommand\p{\mbox{\bf P}}
\newcommand\np{\mbox{\bf NP}}
\newcommand\am{\mbox{\bf AM}}
\newcommand\ma{\mbox{\bf MA}}
\newcommand\distnp{\mbox{\bf DistNP}}
\newcommand\cnp{\mbox{\bf coNP}}
\newcommand\corp{\mbox{\bf coRP}}
\newcommand\sigmatwo{\mbox{\bf $\Sigma_2$}}
\newcommand\sigmathree{\mbox{\bf $\Sigma_3$}}
\newcommand\pitwo{\mbox{\bf $\Pi_2$}}
\newcommand\rp{\mbox{\bf RP}}
\newcommand\zpp{\mbox{\bf ZPP}}
\newcommand\bpp{\mbox{\bf BPP}}
\newcommand\ph{\mbox{\bf PH}}
\newcommand\pspace{\mbox{\bf PSPACE}}
\newcommand\npspace{\mbox{\bf NPSPACE}}
\newcommand\aczero{\mbox{\bf AC$^0$}}
\newcommand\dl{\mbox{\bf L}}
\newcommand\nl{\mbox{\bf NL}}
\newcommand\rl{\mbox{\bf RL}}
\newcommand\sll{\mbox{\bf SL}}
\newcommand\bpl{\mbox{\bf BPL}}
\newcommand\conl{\mbox{\bf coNL}}
\newcommand\sharpp{\mbox{\#{\bf P}}}
\newcommand\parityp{\mbox{$\oplus$ {\bf P}}}
\newcommand\ip{\mbox{\bf IP}}
\newcommand\pcp{\mbox{\bf PCP}}
\newcommand\dtime{\mbox{\bf DTIME}}
\newcommand\ntime{\mbox{\bf NTIME}}
\newcommand\bptime{\mbox{\bf BPTIME}}
\newcommand\dspace{\mbox{\bf SPACE}}
\newcommand\nspace{\mbox{\bf NSPACE}}
\newcommand\cnspace{\mbox{\bf coNSPACE}}
\newcommand\exptime{\mbox{\bf EXPTIME}}
\def\exp{\mbox{\bf EXP}}
\def\ep{\mbox{\bf E}}
\newcommand\nexptime{\mbox{\bf NEXPTIME}}
\newcommand\genclass{\mbox{$\cal C$}}
\newcommand\cogenclass{\mbox{\bf co$\cal C$}}
\newcommand\size{\mbox{\bf SIZE}}

\def\phi{\varphi}

%%Computational problems

\newcommand\sat{\mbox{SAT}}
\newcommand\tsat{\mbox{3SAT}}
\newcommand\tqbf{\mbox{TQBF}}
\newcommand\permp{\mbox{\sc Permanent}}
\def\perm{\operatorname{perm}}

%% Notation for integers, natural numbers, reals, fractions, sets, cardinalities
%%and so on

\newcommand\Z{\mbox{$\mathbb Z$}}
\newcommand\N{\mbox{$\mathbb N$}}
\newcommand\R{\mbox{$\mathbb R$}}
\newcommand\F{\mbox{$\mathcal{F}$}}
\newcommand\B{\{0,1\}}      % boolean alphabet  use in math mode
\newcommand\Bs{\{0,1\}^*}   % B star            use in math mode
\def\H{\{-1,1\}}      
\def\S{{\cal S}}

\newcommand{\bzero}{{\bf 0}}


\newcommand\true{\mbox{\sc True}}
\newcommand\false{\mbox{\sc False}}

\newcommand{\marginlabel}[1]%
{\mbox{}\marginpar{\it{\raggedleft\hspace{0pt}#1}}}
\newcommand\poly{\mbox{poly}}  %usage \poly(n)
\newcommand{\floor}[1]{\lfloor\, $#1$\,\rfloor}
\newcommand{\ceil}[1]{\lceil\, $#1$\,\rceil}
\newcommand{\comp}[1]{\overline{#1}}


%\newcommand\to{\rightarrow} 
\newcommand\xor{\oplus}
\newcommand\bigxor{\bigoplus}
\newcommand{\logred}{\leq_{\log}}
\def\implies{\Rightarrow}


%% probability stuff

\newcommand\E{\mathop\mathbb{E}}
\newcommand\pr{\mathop{\bf {Pr}}}
\newcommand\var{\mathop{\bf Var}}


%% macros to write pseudo-code

\newlength{\pgmtab}  %  \pgmtab is the width of each tab in the
\setlength{\pgmtab}{1em}  %  program environment
 \newenvironment{program}{\renewcommand{\baselinestretch}{1}%
\begin{tabbing}\hspace{0em}\=\hspace{0em}\=%
\hspace{\pgmtab}\=\hspace{\pgmtab}\=\hspace{\pgmtab}\=\hspace{\pgmtab}\=%
\hspace{\pgmtab}\=\hspace{\pgmtab}\=\hspace{\pgmtab}\=\hspace{\pgmtab}\=%
\+\+\kill}{\end{tabbing}\renewcommand{\baselinestretch}{\intl}}
\newcommand {\BEGIN}{{\bf begin\ }}
\newcommand {\ELSE}{{\bf else\ }}
\newcommand {\IF}{{\bf if\ }}
\newcommand {\FOR}{{\bf for\ }}
\newcommand {\TO}{{\bf to\ }}
\newcommand {\DO}{{\bf do\ }}
\newcommand {\WHILE}{{\bf while\ }}
\newcommand {\ACCEPT}{{\bf accept}}
\newcommand {\REJECT}{\mbox{\bf reject}}
\newcommand {\THEN}{\mbox{\bf then\ }}
\newcommand {\END}{{\bf end}}
\newcommand {\RETURN}{\mbox{\bf return\ }}
\newcommand {\HALT}{\mbox{\bf halt}}
\newcommand {\REPEAT}{\mbox{\bf repeat\ }}
\newcommand {\UNTIL}{\mbox{\bf until\ }}
\newcommand {\TRUE}{\mbox{\bf true\ }}
\newcommand {\FALSE}{\mbox{\bf false\ }}
\newcommand {\FORALL}{\mbox{\bf for all\ }}
\newcommand {\DOWNTO}{\mbox{\bf down to\ }}


%%%%%%%%    FALL'05
\newcommand{\coursenum}{CS294}
\newcommand{\coursename}{Pseudorandomness and Combinatorial Constructions}
\newcommand{\courseprof}{Professor Luca Trevisan}



%%%%%%%%%%%%%%%%%%%%%%%%%%%%%%%%%%%%%%%%%%%%%%%%%%%%%%%%%%%%%%%%%%%%%%%%%%%
%%%%%%%%%%%%%%%%%%%%%%%%%%%%%%%%%%%%%%%%%%%%%%%%%%%%%%%%%%%%%%%%%%%%%%%%%%%




\newlength{\tpush}
\setlength{\tpush}{2\headheight}
\addtolength{\tpush}{\headsep}


\newcommand{\handout}[3]{\noindent\vspace*{-\tpush}\newline\parbox{\textwidth}
{U.C. Berkeley  \hfill Handout N#1 \newline
\coursenum : \coursename \hfill #2 \newline
\courseprof \hfill #3 \newline
\mbox{}\hrulefill\mbox{}}\vspace*{1ex}\mbox{}\newline
\bigskip
\begin{center}{\Large\bf Notes for Lecture #1}\end{center}
\bigskip}


\newcommand{\problemset}[2]{\noindent\vspace*{-\tpush}\newline\parbox{\textwidth}
{U.C. Berkeley --- \coursenum : \coursename \hfill Handout #1 \newline
\courseprof \hfill #2 \newline
\mbox{}\hrulefill\mbox{}}\vspace*{1ex}\mbox{}\newline
\bigskip
\bigskip}



















\begin{document}

\handout{Lecture 3}{January 24, 2001}{January 28, 2001}{Scribe: John Leen}

In this lecture 
we will wrap up our discussion of \np-completeness 
by studying a new \np-complete problem, Independent Set, and showing
how 3SAT reduces to it.
Then we will consider non-deterministic models of computation and their
relation to \np.  Finally, we will introduce some space-bounded complexity
classes and examine their relation to the time-bounded classes.

\section{Independent Set is \np-Complete}


We have already seen the proof that SAT and 3SAT are \np-complete.
Now we will look at another \np-complete problem, Independent Set (IS).
We can state problem IS as follows: given a graph $G = (V,E)$ and an
ingeger $k > 0$, is there an independent set in $G$ of size $k$?
(A set of vertices $I \subseteq V$ is {\it independent} if no two
vertices $u, v \in I$ are connected by an edge $(u, v) \in E$.)

There is an optimization version of this problem, which is to find the
largest possible independent set of a given graph, but we are simply
interested in the decision problem IS given above.  We will show that
3SAT reduces to IS, hence IS is \np-complete.

\begin{Thm}
3SAT $\le$ IS.
\end{Thm}
\begin{proof}
Say we have a problem in 3SAT which is a boolean expression $\phi$ with $n$
variables $x_1, x_2, \dots, x_n$, and $m$ clauses.
Construct a graph $G$ with $3m$ vertices, one for each literal (an
occurance of a variable in a clause).  Add edges to connect vertices
corresponding to literals
belonging to the same clause, so that for each clause we have a triangle
of vertices connecting its literal.  Finally, add edges to connect all
pairs of vertices which are negations of each other.

We claim that 3SAT for $\phi$ corresponds to IS for $(G, m)$.  First,
suppose there $is$ an independent set $I$ of size $m$.  Then it must
have exactly one vertex from each triangle, since there are $m$ triangles,
and having two vertices from the same triangle would violate independence.
Let $x_i = true$ if at least one $x_i$ vertex is in $I$, $x_i = false$ if
at least one $\overline{x_i}$ is in $I$, and $x_i = true$ if neither condition
holds.  (The first two conditions cannot both occur for the same $x_i$,
since we constructed
edges between all contradictory pairs of literals.)  By construction,
this satisfies $\phi$, because for each clause we have made at least
one of the three assignments that will satisfy it.

Conversely, if $\phi$ is satisfiable, pick a satisfying assignment of
variables, and define $I$ to contain, for each triangle, one arbitrarily
chosen literal satisfied by the assignment.  (At least one must exist,
or else $\phi$ would not be satisfied!)  This $I$ is independent since
edges only connect literals within the same clause (of which only one
is in $I$) and literals which are negations of each other (of which only
one would be consistent with our satisfying variable assignment).

This proves that $\phi$ can be satisfied if and only if $G$ has an
independent set of size $m$.  $G$ can obviously be constructed from
$\phi$ in polynomial time.  Therefore 3SAT $\le$ IS.
\end{proof}

\section{Nondeterministic Turing Machines}

A Turing machine is usually described by a set of states $Q$, an alphabet
$\Sigma$, and a state transition function
\begin{displaymath}
\delta : Q \times \Sigma \to Q \times \Sigma \times \{ L, S, R \}
\end{displaymath}
which, given the state of the machine and the symbol at the current
position of the tape, tells the machine a new state to go to, a new
symbol to write, and a direction to move in (left, right, or stay in
the same place).  A {\it nondeterministic} Turing machine instead has
a function
\begin{displaymath}
\delta : Q \times \Sigma \to 2^{Q \times \Sigma \times \{ L, S, R \}}
\end{displaymath}
which prescribes multiple possible things for the machine to do next.
We say that a nondeterministic machine $M$ solves a problem $L$ if
for every $x \in L$, there is a possible computation of $M(x)$ leading
to acceptance, and for every $x \notin L$, all computations of $M(x)$
reject.

\begin{Thm}
\np{} is the set of decision problems solvable in polynomial
time by a nondeterministic Turing machine.
\end{Thm}
\begin{proof}
Consider $L \in \np$.  We
know there is an algorithm $V_L(\cdot,\cdot)$ running in polynomial
time $T_L(\cdot)$ and a polynomial $p(\cdot)$ such that
\begin{displaymath}
x \in L \Leftrightarrow \exists y, |y| \le p(|x|) \mbox{ and } V_L(x,y) \mbox{ accepts}
\end{displaymath}
We can construct a machine $M$ that uses nondeterminism to write all
possible strings $y \in \{0,1\}^{p(|x|)}$ to its tape and then simulate
$V_L(x,y)$, essentially testing all possible solutions in parallel.
Thus this nondeterministic machine solves $L$ in polynomial time.

Conversely, if $L$ is solved in polynomial time by a nondeterministic
machine $M$, define $V(\cdot,\cdot)$ that takes as input a value $x$
that would be given as input to $M$ and a description of a computational
path of $M(x)$ leading to acceptance. $V$ accepts if this description
is valid, and runs in polynomial time.  Thus $L \in \np$, establishing
our result.
\end{proof}

\section{Space-Bounded Complexity Classes}

A machine solves a problem using space $s(\cdot)$ if for every
input $x$, the machine outputs the correct answer and uses only
the first $s(|x|)$ cells of the tape.  For a standard Turing machine,
we can't do better than linear space since $x$ itself must be on the
tape!  So we will often consider a machine with multiple tapes: a
read-only ``input'' tape, a read/write ``work'' or ``memory'' tape,
and possibly a write-once ``output'' tape.  Then we can say
the machine uses space $s$ if for input $x$, it uses only the first
$s(|x|)$ cells of the work tape.

We denote by \l{} the set of decision problems solvable in $O(\log n)$ space.
We denote by \pspace{} the set of decision problems solvable in polynomial
space.

\begin{Thm}
If a machine always halts, and uses $s(\cdot)$ space, with $s(n) \ge \log n$,
then it runs in time $2^{O(s(n))}$.
\end{Thm}
\begin{proof}
Call the ``configuration'' of a machine $M$ on input $x$ a description of
the state of $M$, the position of the input tape, and the contents of the work
tape at a given time.  Write down $c_1, c_2, \ldots, c_t$ where $c_i$ is
the configuration at time $i$ and $t$ is the running time of $M(x)$.
No two $c_i$ can be equal, or else the machine would be in a loop, since
the $c_i$ completely describes the present, and therefore the future,
of the computation!  Now the number of {\it possible} configurations is
simply the product of the number of states, the number of positions on
the input tape, and the number of possible contents of the work tape
(which itself depends on the number of allowable positions on the input
tape).
This is
\begin{displaymath}
O(1) \cdot n \cdot |\Sigma|^{s(n)}
= 2^{O(s(n)) + \log n}
= 2^{O(s(n))}
\end{displaymath}
Since we cannot visit a configuration twice during the computation,
the computation must therefore finish in $2^{O(s(n))}$ steps.
\end{proof}

\nl{} is the set of decision problems solvable by a non-deterministic machine
using $O(\log n)$ space.  {\bf NPSPACE} is the set of decision problems solvable
by a non-deterministic machine using polynomial space.

Analogously with
time-bounded complexity classes, we could think that \nl\ is
exactly the set of decision problems that have ``solutions'' that
can verified in log-space. If so, \nl\ would be equal
to \np, since
there is a log-space algorithm $V$ that
verifies solutions to SAT.  However, this is unlikely to be true, because
of the surprising fact that \nl{} is contained in \p{}!  Intuitively,
not all problems with a log-space ``verifier'' can be simulated in
\nl{}, because an {\nl} machine does not have
enough memory to keep track of all the choices that it makes.

\begin{Thm}
$\nl \subseteq \p$.
\end{Thm}
\begin{proof}
Let $L$ be a language in \nl{} and let $M$ be a non-deterministic
log-space machine for $L$.  Consider a computation of $M(x)$.  As before,
at any time there are $2^{O(s(n))} = n^{O(1)}$ possible configurations.
Consider a directed graph in which vertices are configurations and
edges indicate transitions from one state to another which the machine
is allowed to make in a single step (as determined by its $\delta$).
This graph has polynomially many vertices, so in polynomial time
we can do a depth-first search to see whether there is a path from
the initial configuration that eventually leads to acceptance.  This
describes a polynomial-time algorithm for deciding $L$, so we're done.
\end{proof}

\end {document}