\documentclass[11pt]{article}
\usepackage
[pdftex,pagebackref,letterpaper=true,colorlinks=true,pdfpagemode=none,urlcolor=blue,linkcolor=blue,citecolor=blue,pdfstartview=FitH]{hyperref}

\usepackage{amsmath,amsfonts}
\usepackage{graphicx}

\bibliographystyle{alpha}

\setlength{\oddsidemargin}{0pt}
\setlength{\evensidemargin}{0pt}
\setlength{\textwidth}{6.0in}
\setlength{\topmargin}{0in}
\setlength{\textheight}{8.5in}

\setlength{\parindent}{0in}
\setlength{\parskip}{5pt}


%%%%%%%%% For wordpress conversion

\def\more{}

\usepackage{ulem}
\def\em{\it}
\let\emph=\it


\newcommand{\coursenum}{CS172}
\newcommand{\coursename}{Automata, Computability and Complexity}
\newcommand{\courseprof}{Professor Luca Trevisan}

%       Usage: \ptitle{title}{dateout}
\newcommand{\ptitle}[2]{\noindent\parbox{\textwidth}
{U.C. Berkeley --- \coursenum : \coursename \hfill #1 \newline
\courseprof \hfill #2 \newline
\mbox{}\hrulefill\mbox{}}\vspace*{1ex}\mbox{}\newline
\bigskip
\begin{center}{\Large\bf #1}\end{center}
\bigskip}


%       Usage: \handout{title}{datelec}{dateout}{scribe}
\newcommand{\handout}[2]{\thispagestyle{empty}
 \markboth{Notes for Lecture #1}{Notes for Lecture #1}
 \pagestyle{myheadings}\htitle{#1}{#2}}

%       Usage: \pset{title}{dateout}
\newcommand{\pset}[2]{\thispagestyle{empty}
 \markboth{#1 --- #2}{#1 --- #2}
 \pagestyle{myheadings}\ptitle{#1}{#2}}


\begin{document}

\handout{Lecture 2}{January 22, 2001}{January 23, 2001}{}

\bibliographystyle{alpha}

This lecture is devoted to the proof that SAT, a decision
problem from logics, is \np-complete. This result
is due to Cook \cite{C71}. Cook introduced the notion
of reductions and \np-completeness, recognized the
relevance of the notion of \np-completeness, and proved
the \np-completeness of SAT and of a few other problems.
Independently, Levin \cite{L73} also developed the notions
of reductions and \np-completeness, and proved
the \np-completeness of some natural problems. The
\np-completeness of SAT is often referred to as
{\em Cook's theorem} or {\em Cook-Levin's theorem}.
Karp \cite{K72}  proved the \np-completeness
of several problems, including several optimization
problems. Karp's work pointed out the potential of
\np-completenss as a way of explaining the hardness of
an extreme variety of problems. This promise was quickly
fulfilled, and \np-completeness results came by the
hundreds during the 70s.

\section{The Problem SAT}

In SAT (that stands for {\em CNF-satisfiability}) we are
given Boolean variables $x_1,x_2,\ldots,x_n$ and a 
Boolean formula $\phi$ involving such variables; the formula
is given in a particular format called {\em conjunctive
normal form}, that we will explain in a moment.
The question is whether there is a way to assing Boolean
($\true$ / $\false$) values to the variables so that the
formula is satisfied.

To complete the description of the problem we need
to explain what is a Boolean formula in conjunctive
normal form. First of all, Boolean formulas are
constructed starting from variables and
applying the operators $\vee$ (that stands for OR),
$\wedge$ (that stands for AND) and $\neg$ (that stands
for NOT).

The operators work in the way that one expects:
$\neg x$ is $\true$ if and only if $x$ is $\false$; $x\wedge y$
is $\true$ if and only if both $x$ and $y$ are $\true$;
$x \vee y$ is $\true$ if and only at least one of $x$ or $y$
is $\true$.

So, for example, the expression $\neg x \wedge (x \vee y)$
can be satisfied by setting $x$ to $\false$ and $y$ to $\true$,
while the expression $x \wedge (\neg x \vee y) \wedge \neg y$
is impossible to satisfy.

A {\em literal} is a variable or the negation of a variable,
so for example $\neg x_7$ is a literal and so is $x_3$.
A {\em clause} is formed by taking one or more literals
and connecting them with a OR, so for example $(x_2 \vee \neg x_4
\vee x_5)$ is a clause, and so is $(x_3)$. A {\em formula
in conjunctive normal form} is the AND of clauses. For
example
\[ (x_3 \vee \neg x_4) \wedge (x_1) \wedge (\neg x_3 \vee x_2) \]
is a formula in conjunctive normal form (from now on, we will
just say ``CNF formula'' or ``formula''). 
Note that the above formula is satisfiable,
and, for example, it is satisfied by setting all the variables
to $\true$ (there are also other possible assignments of values to
the variables that would satisfy the formula).

On the other hand, the formula
\[ x \wedge (\neg x \vee y) \wedge \neg y \]
is not satisfiable, as it has already been observed.


\section{Intuition for the Reduction}

From now on, it will be convenient to think of
Boolean variables as taking values in $\B$, with
the convention that $0$ stands for $\false$ and $1$
stands for $\true$.

It may seem that CNF formulas can only specify very simple
Boolean relations, however the following result holds.

\begin {Lem}\label{lm:bool}
For every function $f: \B^k \to \B$ there is a CNF formula
$\phi$ over variables $x_1,\ldots,x_k$ such that an
assignment $b_1,\ldots,b_k$ to $x_1,\ldots,x_k$ satisfies
$\phi$ if and only $f(b_1,\ldots,b_k)=1$. Furthermore,
the size of $\phi$ is at most $k2^k$.
\end {Lem}

So it is possible to represent every function using
a CNF formula, although the formula promised by
the above lemma could be exponentially big in the
number of variables. If we could guarantee the
formula to be small for all efficiently computable $f$,
then the \np-completeness of SAT would be quite easy
to establish.

Let $L$ be an \np\ decision problem. Then, by definition,
there is an algorithm $V_L$ and polynomials $T$ and $p$
such that for every $x$, we have that $x\in L$ if and
only if there is a $y$, $|y| \leq p(|x|)$ such that
$V_L(x,y)$ accepts; furthermore $V_L(x,y)$ runs
at most in time $T(|x|+|y|)$. Now, fix an $x$,
call $l = p(|x|)$, and consider the function
$f_x :\B^l \to \B$ such that $f_x(y)=1$ if and only
if $V_L(x,y)$ accepts. Suppose we could produce a CNF formula
$\phi_x$ over $l$ variables such that $\phi_x$ is 
satisfied only by assignments that correspond to
a $y$ such that $f_x(y)=1$. Then $\phi_x$ is satisfiable
if and only if $x\in L$, and the transformation
mapping $x$ in $\phi_x$ would be a reduction
from $L$ to SAT. Since $L$ was arbitrary, this
argument would prove the $\np$-hardness of SAT.

Unfortunately, the function $f_x$ could be such that
there is no equivalent polynomial-size CNF formula,
so it may be impossible to implement this strategy
in polynomial time. In fact, even very simple
functions have no short CNF formula representation,
and the above strategy cannot possibly work, as shown
by the next example.

\begin {Lem}
Let $\phi$ be a formula over $k$ variables, such that
$\phi$ is satisfied precisely by those assignments where
an odd number of variables takes value 1. Then $\phi$
has at least $2^{k-1}$ clauses.
\end{Lem}

\section{The Reduction}

Fortunately, there is a way to implement the general idea
given in the previous section; unfortunately, there is
some technical work to do.

Let $L$ be an \np\ decision problem, and let $V(\cdot,\cdot)$
be the algorithm and $T$ and $p$ be the polynomials as above.
We assume that $V$ is implemented as a single-tape Turing
machine $M$ with semi-infinite tape.

Let us also fix an input $x$, and let $l = p(|x|)$ and
$t=T(|x|+l)$. 

Our goal is to construct, in time polynomial
in $|x|$, a CNF formula $\phi_x$ such that $\phi_x$ is
satisfiable if and only if $x\in L$, that is, if and only
there is a $y$, $|y| \leq l$ such that $M(x,y)$ accepts
within $t$ steps. 

We will construct $\phi_x$ so that its variables not
only represent a possible choice of $y$, but also give
a complete representation of the whole computation
of $M(x,y)$. Let $Q$ be the set of states of $M$,
and let $\Sigma$ be the alphabet of $M$. Since
we want to represent computations of $M$ that take
time at most $t$, we know that only the first $t$
entries of the tape are relevant for the description.
The global state (also called a {\em configuration})
of the machine can be described by giving the
position of the head on the tape (in one of the possible
$t$ positions), the current state of the machine
(one of the $|Q|$ possible ones), and giving the
content of the relevant part of the tape (a
string in $\Sigma^t$). We will choose a more
redundant representation, that will be easier to
deal with. For each cell of the tape, we
specify a pair $(q,a)$, where $q\in Q \cup \{ \square \}$
is the state of the machine, if the head is on
that particular cell, or the special symbol $\square$
otherwise, and $a$ is the content of the cell. So
the configuration is represented by a string in
$(\Sigma \times (Q \cup \{ \square\}))^t$.
A computation of length $t$ can be represented by
a sequence of $t+1$ such configurations. Such a
$t \times (t+1)$ table of elements of 
$\Sigma \times (Q \cup \{ \square\})$
will be called the {\em tableau} of a computation.

The variables of the formula $\phi_x$ are meant
to represent a tableau. For every $i=0,\ldots,t$,
$j=1,\ldots,t$, $q\in Q \cup \{ \square \}$,  $a\in \Sigma$,
we have a variable $z_{i,j,q,a}$. The intended
meaning of the variable is that the variable is true
if and only if the tableau contains the pair $(q,a)$
in the entry corresponding to time $i$ and position $j$
on the tape. Overall, we have $t\times(t+1)\times |\Sigma|
\times(|Q|+1)$ variables.

The clauses of $\phi_x$, that we will describe in a moment,
enforce the property that every assignment to $z_{i,j,q,a}$
that satisfies $\phi_x$ encodes the tableau of a computation
$M(x,y)$ where $|y| \leq l$, and $M$ accepts at the end.

First of all, we need to enforce that the variables
are giving a consistent description of the tableau,
that is, that for each fixed $i$ and $j$, there is
exactly one pair $(q,a)$ such that $z_{i,j,q,a}$ is true.
This is enforced by the clauses
\[ (\bigvee_{q,a} z_{i,j,q,a}) \mbox{ for all } i=0,\ldots,t, \ j = 1,\ldots,t
 \]
and the clauses
\[ (\neg z_{i,j,q,a} \vee \neg z_{i,j,q',a'}) \mbox{ for all }
i=0,\ldots,t,\ j=1,\ldots,t, \ (q,a) \neq (q',a')\]

We also want to make sure that, in every line, exactly one
cell contains a pair $(q,a)$ where $q\neq \square$. We will
use the clauses
\[ (\bigvee_{q\neq \square,a,j} z_{i,j,q,a}) \mbox{ for all } i \]
and the clauses
\[ (\neg z_{i,j,q,a} \vee \neg z_{i,j',q',a'}) \mbox{ for all }
i, j\neq j', q,q' \in Q, a,a'\in \Sigma\]



Next, we want to enforce that the first line corresponds the
a starting configuration where the tape
contains $(x,y)$ for some $y$ of length $l$. Let $q_0$
be the initial state, let $n=|x|$, and let 
$x = (x_1,\ldots,x_n)$. We will use the clauses

\[ z_{0,1,q_0,x_1} \]
\[ z_{0,j,\square,x_i} \mbox{ for all }j=2,\ldots,n \]
\[ (z_{0,j,\square,0} \vee z_{0,j,\square,1}) \mbox{ for all } 
j=n+1,\ldots,n+l
 \]
\[ z_{0,j,\square,\square} \mbox{ for all } j=n+l+1,\ldots,t \]

Then, we want to enforce that each line of the tableau is consistent
with the previous line. Note that the content of the entry $(i,j)$
of the tableau only depends on the entries $(i-1,j-1)$, $(i-1,j)$
and $(i-1,j+1)$ of the tableau (for $i\geq 2$). We can then
use Lemma \ref{lm:bool} to claim that
for each $i,j$, $i\geq 2$, there is a CNF formula $\phi_{i,j}$
over the variables of the form $z_{i,j,q,a}$, $z_{i-1,j-1,q,a}$,
$z_{i-1,j,q,a}$ and $z_{i-1,j-1,q,a}$ that is satisfied if and
only if the variables encode a transition that is consistent
with behaviour of $M$. Each such formula $\phi_{i,j}$ relates
to $4\times (|Q|+1)\times |\Sigma| = O(1)$ variables,
and so it is of size $O(1)$. We can construct all those
formulas, and then take their AND, together with all the clauses
described above.

Finally, we add the clause $\vee_{j,a} z_{t,j,q_A,a}$, where
$q_A$ is the accepting state of $M$.

The resulting formula is $\phi_x$. Looking back at
the description of the clauses in $\phi_x$, the construction
of $\phi_x$ can be done in time polynomial in $|x|$.



We claim that $\phi_x$ is satisfiable if and only if $x\in L$;
this proves that $L$ is reducible to SAT.

Suppose $x\in L$, then there is $y$, $|y|\leq l$ such that
$M(x,y)$ accepts in at most $t$ steps. Write down the tableau
of the computation of $M(x,y)$, then compute the corresponding
assignment to variables $z_{i,j,q,a}$. One can verify that such
an assignment satisfies all the clauses described above.

Suppose we have an assignment of values $b_{i,j,q,a}$ to
variables $z_{i,j,q,a}$ that satisfies $\phi_x$. 
Construct the tableau corresponding to the values $b_{i,j,q,a}$.
Since all clauses of $\phi_x$ are satisfied, the tableau
describes a computation $M(x,y)$, for some $y$ of length $l$,
such that $M(x,y)$ accepts in at most $t$ steps. Then
it must be the case that $x\in L$.

Since $L$ was arbitrary, we deduce that SAT is \np-hard.
It's easy to see that SAT is in \np, so we proved that
SAT is \np-complete.

\section{The NP-Completeness of 3SAT}

3SAT is the restriction of SAT to formulas where every clause
is the OR of precisely 3 literals. A CNF formula where
every clause contains exactly 3 literals is also called
a 3CNF formula.

We give a reduction from SAT to 3SAT, thereby showing that
3SAT is NP-complete.

Let $\phi$ be a CNF formula, we want to create
a new formula $\phi_3$ such that $\phi$ is satisfiable
if and only $\phi_3$ is satisfiable; furthermore $\phi_3$
contains only clauses with exactly 3 literals.

Note that it is not possible to construct $\phi_3$ so that
it has the same set of variables as $\phi$, and is satisfied
precisely by the same assignments. Consider for example
the case where $\phi$ contains only the clause $(x_1 \vee x_2
\vee x_3 \vee x_4)$. There is no 3CNF formula over variables
$x_1,x_2,x_3,x_4$ that has the same set of satisfying assignments
as $\phi$.

Starting from an arbitrary CNF formula $\phi$,
we will construct $\phi_3$ by substitution each clause in
$\phi$ that contains 1, 2, 4 or more literals by a small 3CNF
formula, involving new variables. Each substitution will
preserve satisfiability.

A clause of the form $(x)$ is replaced by 
\[ (x \vee y_1 \vee y_2) \wedge  (x \vee \neg y_1 \vee y_2)
\wedge  (x \vee y_1 \vee \neg y_2) \wedge  (x \vee \neg y_1 \vee \neg y_2) \]
where $y_1,y_2$ are new variables.

A clause of the form $(x_1 \vee x_2)$ is replaces by
\[ (x_1 \vee x_2 \vee y) \wedge (x_1 \vee x_2 \vee \neg y)\]
where $y$ is a new variable.

A clause of the form $(x_1 \vee \cdots \vee x_k$, $k\geq 4$ is
replaced by
\[ (x_1 \vee x_2 \vee y_1) \wedge (\neg y_1 \vee x_3 \vee y_2)
\wedge \cdots \wedge (\neg y_{k-3} \vee x_{k-1} \vee x_k) \]
where $y_1,\ldots,y_{k-3}$ are new variables.

One should check that each of these substitutions do indeed
preserve satisfiability, and so that the combination of all these
substitutions give a reduction that produces a formula $\phi_3$ that
is satisfiable if and only if $\phi$ is satisfiable.

\bibliography{macros,complexity}

\end {document}