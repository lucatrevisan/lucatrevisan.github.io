\documentclass[11pt]{article}
\usepackage
[pdftex,pagebackref,letterpaper=true,colorlinks=true,pdfpagemode=none,urlcolor=blue,linkcolor=blue,citecolor=blue,pdfstartview=FitH]{hyperref}

\usepackage{amsmath,amsfonts}
\usepackage{graphicx}

\bibliographystyle{alpha}

\setlength{\oddsidemargin}{0pt}
\setlength{\evensidemargin}{0pt}
\setlength{\textwidth}{6.0in}
\setlength{\topmargin}{0in}
\setlength{\textheight}{8.5in}

\setlength{\parindent}{0in}
\setlength{\parskip}{5pt}


%%%%%%%%% For wordpress conversion

\def\more{}

\usepackage{ulem}
\def\em{\it}
\let\emph=\it


\newcommand{\coursenum}{CS172}
\newcommand{\coursename}{Automata, Computability and Complexity}
\newcommand{\courseprof}{Professor Luca Trevisan}

%       Usage: \ptitle{title}{dateout}
\newcommand{\ptitle}[2]{\noindent\parbox{\textwidth}
{U.C. Berkeley --- \coursenum : \coursename \hfill #1 \newline
\courseprof \hfill #2 \newline
\mbox{}\hrulefill\mbox{}}\vspace*{1ex}\mbox{}\newline
\bigskip
\begin{center}{\Large\bf #1}\end{center}
\bigskip}


%       Usage: \handout{title}{datelec}{dateout}{scribe}
\newcommand{\handout}[2]{\thispagestyle{empty}
 \markboth{Notes for Lecture #1}{Notes for Lecture #1}
 \pagestyle{myheadings}\htitle{#1}{#2}}

%       Usage: \pset{title}{dateout}
\newcommand{\pset}[2]{\thispagestyle{empty}
 \markboth{#1 --- #2}{#1 --- #2}
 \pagestyle{myheadings}\ptitle{#1}{#2}}


\newenvironment{proofidea}{\noindent {\sc Proof Idea:}}{\medskip} 
\begin{document}

\handout{Lecture 4}{January 29, 2001%date of lecture
}{\today 
%date of last revision
}
{Allison Coates 
%name of scribe
}

\bibliographystyle{alpha}


Today's lecture continues the discussion of space complexity. We
examine $\nl$, the set of languages decidable in $O(\log {n})$ space on a
nondeterministic Turing machine, introduce log-space reducibility and
define nl-completeness. Finally, we discuss Savitch's theorem, and
state the main result to be proved next time: $\nl = \conl$.

\section{Reductions in \nl}

Last time, we introduced \nl, the class of languages decidable in
logarithmic space on a nondeterministic Turing Machine: \nl =
\nspace{$(\log{n})$}. Now we would like to introduce a notion
of completeness in \nl analogous to the notion of completeness
that we have introduced and studied for the class \np. A first
observation is that, in order to have a meaningful notion of
completeness in \nl, we cannot use polynomial-time reductions,
otherwise any \nl problem having at least a YES instance
and at least a NO instance would be trivially \nl-complete.
To get a more interesting notion of \nl completeness we need
to turn to weaker reductions.
In particular, we define {\em log space} reductions as follows:
\begin{Def}
    Let $A$ and $B$ be decision problems. We say $A$ is log space
    reducible to $B$, $A \logred B$, if $\exists$ a function $f$
    computable in log space such that $x \in A$ iff $f(x) \in B$, and $B
    \in L$.
\end{Def}

\begin{Thm} If $B \in \dl$, and $A \logred B$, then $A \in \dl$. 
\end{Thm}
\begin{proof} 
    We consider the concatenation of two machines: $M_f$ to compute $f$,
    and $M_B$ to solve $B$. If our resource bound was polynomial time, 
    then we would use $M_f(x)$ to compute
    $f(x)$, and then run $M_B$ on $f(x)$. The composition of the
    two procedures would given an algorithm for $A$, and if both
    procedures run in polynomial time then their composition is
    also polynomial time. To prove the theorem, however, we have
    to show that if $M_f$ and $M_B$ are log space machines, then
    their composition can also be computed in log space.
    
    Recall the definition of a Turing machine $M$ that has a log space
    complexity bound: $M$ has one read-only input tape, one write-only
    output tape, and uses a log space work tape.  A
    naive implementation of the composition of $M_f$ and $M_B$
    would be to compute $f(x)$, and then run $M_B$ on input $f(x)$;
    however $f(x)$ needs to be stored on the work tape, and this
    implementation does not produce a log space machine.
    Instead we modify $M_f$ so that on input $x$ and $i$ it
    returns the $i$-th bit of $f(x)$ (this computation can still
    be carried out in logarithmic space). Then we run a 
    simulation of the computation of $M_B(f(x))$ by using
    the modified $M_f$ as an ``oracle'' to tell us the value
    of specified positions of $f(x)$. In order to simulate $M_B(f(x))$
    we only need to know the content of one position of $f(x)$ at a
    time, so the simulation can be carried with a total of
    $O(\log |x|)$ bits of work space.
\end{proof}

Using the same proof technique, we can show the following:
\begin{Thm}
    if $A \logred B, B \logred C$, then $ A \logred C$.
\end{Thm}
\section{$\nl$ Completeness}
Armed with a definition of log space reducible, we can define
\nl-completeness.
\begin{Def}
$A$ is $\nl$-hard if $\forall\, B  \in \nl, B \logred A$. 
$A$ is $\nl$-complete if 
$A \in \nl$  and $A$ is $\nl$-hard.
\end{Def}
We now introduce a problem STCONN (s,t-connectivity) that we will show
is $\nl$-complete. In  STCONN, given in input  a directed graph
$G(V,E)$ and two vertices $s, t \in V$, we want to determine if 
there is  a
directed path from $s $ to $t$.

\begin{Thm} STCONN is $\nl$-complete.
\end{Thm}

\begin{proof}
\begin{enumerate}
\item STCONN $\in \nl$.

    On input {\it G(V,E), s,t}, set $p$ to $s$.
    For $i = 1$ to $|V|$, nondeterminsitically, choose a neighboring
    vertex $v$ of $p$.  Set $p = v$. If $p = t$, accept and halt.
    Reject and halt if the end of the {\em for} loop is reached. 
    The algorithm only requires
    $O(\log{n})$ space.

\item STCONN is $\nl$-hard.
    
    Let $A \in \nl, and let M_A$ be a non-deterministic logarithmic
    space Turing Machine for $A$.  On input $x$,
    construct a directed graph $G$ with one vertex for each
    configuration of $M(x)$, and an additional vertex $t$.  Add edges
    $(c_i, c_j)$ if $M(x)$ can move in one step from $c_i$ to $c_j$.
    Add edges $(c ,t)$ from every configuration that is accepting, and
    let $s$ be the start configuration.  $M$ accepts $x$ iff some path
    from $s$ to $t$ exists in $G$.  The above graph can be
    constructed from $x$ in log space, because listing all nodes
    requires $O(\log{n})$ space, and testing valid edges is also easy.
\end{enumerate}
\end{proof}



\section{Savitch's Theorem}
What kinds of tradeoffs are there between memory and time?  STCONN can
be solved deterministically in linear time and linear space, using
depth-first-search.  Is there some sense in which this is optimal?
Nondeterministically, we can search using less than linear space.  Can
searching be done deterministically in less than linear space?

We will use Savitch's Theorem to show that STCONN can be solved
deterministically in $O(\log^{2}{n})$, and that every $\nl$ problem
can be solved deterministically in $O(\log^{2}{n})$ space.  In
general, if $A$ is a problem that can be solved nondeterministically
with space $s(n) \geq \log{n}$, then it can be solved
deterministically with $O(s^2(n))$space.

\begin{Thm}
Every problem in $\nl$ can be solved deterministically in $O(\log^{2}{n})$ space.
\end{Thm}
\begin{proof}
    Consider a graph $G(V,E)$, and vertices $s, t$. We define a
    recursive function REACH$(u,v,k)$ that accepts and halts iff $v$
    can be reached from $u$ in $\leq k$ steps. If $k = 1$, then REACH
    accepts iff $(u,v)$ is an edge. If $k \geq 2 , \forall w \in V-
    \{u, v\}$, compute REACH$(u,w, \floor{k/2}$) and REACH$(w,v,
    \ceil{k/2}$). If both accept and halt, accept.  Else, reject.
    
    Let $S(k)$ be the worst-case space use of REACH$(\cdot,\cdot,k)$.
    The space required for the base case $S(1)$ is a counter for
    tracking the edge, so $S(1) = O(\log n)$. 
    In general, $S(k) = O(\log n ) + S(k/2)$ for
    calls to REACH and for tracking $w$.  So, $S(k) = O(\log{k}
    *\log{n})$. Since $ k \leq n$, the worst-case space use
    of REACH is $O(\log^2{n})$.
\end{proof}

Comparing to depth-first-search, we find that we are exponentially
better in space requirements, but we are no longer polynomial in time.

Examining the time required, if we let $t(k)$ be
the worst-case time used by REACH$(\cdot,\cdot,k)$,
we see $t(1) = O(n +m)$, and $t(k) =
n(2*T(k/2))$, which solves to $t(k) = n^{O(\log{k})} = O(n^{O(\log{n})})$, 
which is
super-polynomial.  Savitch's algorithm is still the one with the
 best known space
bound. No known algorithm achieves polynomial log space and polynomial
time simultaneously. However, there are interesting results for STCONN
in undirected graphs. There exists an algorithm running in polynomial
time and $O(\log^{2}{n})$ space (but the polynomial has very high
degree), due to Nisan \cite{N94}. 
There is also an algorithm that has  $O(\log^{4/3}{n})$ space
complexity and superpolynomial time complexity, due to
Armoni, Ta-Shma, Nisan and Wigderson \cite{ATWZ97}, improving
on a previous algorithm by Nisan, Szemeredy and Wigderson \cite{NSW92}.

 
\begin{Thm}[Savitch's Theorem] For every function $s(n)$ computable
in space $O(s(n))$,
$\nspace(s) = \dspace(O(s^2))$
\end{Thm}

\begin{proof}
We begin with a nondeterministic machine $M$, which on input $x$ uses
$s(|x|)$ space. We define $REACH (c_i, c_j, k)$, as above, which accepts
and halts iff $M(x)$ can go from $c_i$ to $c_j$ in $\leq k$ steps.  We
compute $REACH(c_0, c_{\hbox{acc}}, 2^O(s|x|))$ for all accepting
configurations $c_{\hbox{acc}}$.  
If there is a call of REACH which accepts and halts,
then $M$ accepts. Else, $M$ rejects.  If REACH accepts and halts,
it will do so in $\leq 2^{O(|x|)}$ steps.

Let $S_R(k)$ be the worst-case space used by  REACH$(\cdot,\cdot,k)$:
$S_R(1) = O(s(n)), S_R(k) = O(s(n) ) + S_R(k/2)$. This solves
$S_R =s(n)* \log{k }$, and, since $k = 2^O(s(n))$,
we have $S_R = O(s^2(n))$.
\end {proof}

\section{\conl}

We now show interesting results for $\nl$ and $\conl$.  Let
$\comp{A}$ denote the complement of $A$ --- that is, if $A$ is a set of
strings for which $M_A$ accepted, then $\comp{A}$ consists of exactly
those strings for which $M_A$ did not accept. For
a complexity class \genclass, we define the {\em complement class}
\cogenclass as follows:  a decision problem $A
\in \cogenclass$  iff $\comp{A} \in \genclass$.

Observe that, for deterministic classes, the complement of
a class is the class itself. However, for nondeterministic
classes, complementarity involves an exchange of the
quantification: one goes from $\exists$ something satisfying such a
condition to $\forall$ cases, there is nothing satisfying that
condition.  If $A\in \np$, $X \in A \Leftrightarrow y, |y| \leq p(x).
V(x,y)$ accepts. If $A \in \cnp$, $ x \in A
\Leftrightarrow \forall\, y, |y| \leq p(x) . V(x,y)$ rejects.
It considered very unlikely that $\np = \cnp$, however
the situation is different for space-bounded nondeterministic
classes. The following theorem is due to Immerman \cite{I88}
and Sezelepscenyi \cite{S88:nl}, and we will prove it next time.

\begin{Thm}
$\nl= \conl$
\end{Thm}

For the time being, let us see how the theorem reduces
to finding a nondeterminisitc log space procedure for
the complement of the STCONN problem.

\begin{Lem}
Suppose $\comp{STCONN} \in \nl$, then $\nl = \conl$.
\end{Lem}

\begin{proof}
Consider
a generic problem $A$ in $\conl$.
$\comp{A} \in \nl$, so $\comp{A} \logred STCONN$.
This means that there is a function $f$ computable
in logarithmic space such that $x \in \comp A
 \Leftrightarrow f( x) \in STCONN$. But then we also have
$x \in  A \Leftrightarrow  f(x) \in \comp{STCONN}$,
and so $A \logred \comp {STCONN}$. Since, by assumption,
we have $\comp {STCONN} \in \nl$, and $\nl$ is closed under
logarithmic space reductions, we conclude $A\in \nl$,
and since $A$ was arbitrary, we have $\conl \subseteq \nl$.

Take now any problem in $\nl$:
$A \in \nl$, hence $\comp{A} \in \conl$
but by above, $\comp{A} \in \nl$, $A \in \conl$, 
so $\nl \subseteq \conl$.
\end{proof}

\bibliography{macros,complexity,random}


\end{document}

