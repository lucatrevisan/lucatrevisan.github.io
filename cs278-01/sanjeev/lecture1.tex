\documentclass[11pt]{article}
\usepackage
[pdftex,pagebackref,letterpaper=true,colorlinks=true,pdfpagemode=none,urlcolor=blue,linkcolor=blue,citecolor=blue,pdfstartview=FitH]{hyperref}

\usepackage{amsmath,amsfonts}
\usepackage{graphicx}

\bibliographystyle{alpha}

\setlength{\oddsidemargin}{0pt}
\setlength{\evensidemargin}{0pt}
\setlength{\textwidth}{6.0in}
\setlength{\topmargin}{0in}
\setlength{\textheight}{8.5in}

\setlength{\parindent}{0in}
\setlength{\parskip}{5pt}


%%%%%%%%% For wordpress conversion

\def\more{}

\usepackage{ulem}
\def\em{\it}
\let\emph=\it



%Formatting


\setlength{\topmargin}{0in}
\setlength{\oddsidemargin}{0 in}
\setlength{\evensidemargin}{0 in}
\setlength{\textwidth}{6.5 in}
\setlength{\textheight}{8.5 in}
\setlength{\headsep}{0.75 in}
\setlength{\parindent}{0 in}
\setlength{\parskip}{0.05 in}

%Theorems, proofs, etc.

\newtheorem{theorem}{Theorem}
\newtheorem{lemma}[theorem]{Lemma}
\newtheorem{corollary}[theorem]{Corollary}
\newtheorem{proposition}[theorem]{Proposition}
\newtheorem{claim}[theorem]{Claim}
\newtheorem{exercise}{Exercise}
\newtheorem{conjecture}{Conjecture}
\newtheorem{example}{Example}
\newtheorem{remark}{Remark}
\newtheorem{definition}[theorem]{Definition}
\newenvironment{proof}{\noindent {\sc Proof:}}{$\Box$\medskip} 
\newtheorem{question}{Open Question}
\newtheorem{problem}{Problem}



%% Macros for complexity theory notation
%% Complexity classes
\newcommand\p{\mbox{\bf P}}
\newcommand\np{\mbox{\bf NP}}
\newcommand\am{\mbox{\bf AM}}
\newcommand\ma{\mbox{\bf MA}}
\newcommand\distnp{\mbox{\bf DistNP}}
\newcommand\cnp{\mbox{\bf coNP}}
\newcommand\corp{\mbox{\bf coRP}}
\newcommand\sigmatwo{\mbox{\bf $\Sigma_2$}}
\newcommand\sigmathree{\mbox{\bf $\Sigma_3$}}
\newcommand\pitwo{\mbox{\bf $\Pi_2$}}
\newcommand\rp{\mbox{\bf RP}}
\newcommand\zpp{\mbox{\bf ZPP}}
\newcommand\bpp{\mbox{\bf BPP}}
\newcommand\ph{\mbox{\bf PH}}
\newcommand\pspace{\mbox{\bf PSPACE}}
\newcommand\npspace{\mbox{\bf NPSPACE}}
\newcommand\aczero{\mbox{\bf AC$^0$}}
\newcommand\dl{\mbox{\bf L}}
\newcommand\nl{\mbox{\bf NL}}
\newcommand\rl{\mbox{\bf RL}}
\newcommand\sll{\mbox{\bf SL}}
\newcommand\bpl{\mbox{\bf BPL}}
\newcommand\conl{\mbox{\bf coNL}}
\newcommand\sharpp{\mbox{\#{\bf P}}}
\newcommand\parityp{\mbox{$\oplus$ {\bf P}}}
\newcommand\ip{\mbox{\bf IP}}
\newcommand\pcp{\mbox{\bf PCP}}
\newcommand\dtime{\mbox{\bf DTIME}}
\newcommand\ntime{\mbox{\bf NTIME}}
\newcommand\bptime{\mbox{\bf BPTIME}}
\newcommand\dspace{\mbox{\bf SPACE}}
\newcommand\nspace{\mbox{\bf NSPACE}}
\newcommand\cnspace{\mbox{\bf coNSPACE}}
\newcommand\exptime{\mbox{\bf EXPTIME}}
\def\exp{\mbox{\bf EXP}}
\def\ep{\mbox{\bf E}}
\newcommand\nexptime{\mbox{\bf NEXPTIME}}
\newcommand\genclass{\mbox{$\cal C$}}
\newcommand\cogenclass{\mbox{\bf co$\cal C$}}
\newcommand\size{\mbox{\bf SIZE}}

\def\phi{\varphi}

%%Computational problems

\newcommand\sat{\mbox{SAT}}
\newcommand\tsat{\mbox{3SAT}}
\newcommand\tqbf{\mbox{TQBF}}
\newcommand\permp{\mbox{\sc Permanent}}
\def\perm{\operatorname{perm}}

%% Notation for integers, natural numbers, reals, fractions, sets, cardinalities
%%and so on

\newcommand\Z{\mbox{$\mathbb Z$}}
\newcommand\N{\mbox{$\mathbb N$}}
\newcommand\R{\mbox{$\mathbb R$}}
\newcommand\F{\mbox{$\mathcal{F}$}}
\newcommand\B{\{0,1\}}      % boolean alphabet  use in math mode
\newcommand\Bs{\{0,1\}^*}   % B star            use in math mode
\def\H{\{-1,1\}}      
\def\S{{\cal S}}

\newcommand{\bzero}{{\bf 0}}


\newcommand\true{\mbox{\sc True}}
\newcommand\false{\mbox{\sc False}}

\newcommand{\marginlabel}[1]%
{\mbox{}\marginpar{\it{\raggedleft\hspace{0pt}#1}}}
\newcommand\poly{\mbox{poly}}  %usage \poly(n)
\newcommand{\floor}[1]{\lfloor\, $#1$\,\rfloor}
\newcommand{\ceil}[1]{\lceil\, $#1$\,\rceil}
\newcommand{\comp}[1]{\overline{#1}}


%\newcommand\to{\rightarrow} 
\newcommand\xor{\oplus}
\newcommand\bigxor{\bigoplus}
\newcommand{\logred}{\leq_{\log}}
\def\implies{\Rightarrow}


%% probability stuff

\newcommand\E{\mathop\mathbb{E}}
\newcommand\pr{\mathop{\bf {Pr}}}
\newcommand\var{\mathop{\bf Var}}


%% macros to write pseudo-code

\newlength{\pgmtab}  %  \pgmtab is the width of each tab in the
\setlength{\pgmtab}{1em}  %  program environment
 \newenvironment{program}{\renewcommand{\baselinestretch}{1}%
\begin{tabbing}\hspace{0em}\=\hspace{0em}\=%
\hspace{\pgmtab}\=\hspace{\pgmtab}\=\hspace{\pgmtab}\=\hspace{\pgmtab}\=%
\hspace{\pgmtab}\=\hspace{\pgmtab}\=\hspace{\pgmtab}\=\hspace{\pgmtab}\=%
\+\+\kill}{\end{tabbing}\renewcommand{\baselinestretch}{\intl}}
\newcommand {\BEGIN}{{\bf begin\ }}
\newcommand {\ELSE}{{\bf else\ }}
\newcommand {\IF}{{\bf if\ }}
\newcommand {\FOR}{{\bf for\ }}
\newcommand {\TO}{{\bf to\ }}
\newcommand {\DO}{{\bf do\ }}
\newcommand {\WHILE}{{\bf while\ }}
\newcommand {\ACCEPT}{{\bf accept}}
\newcommand {\REJECT}{\mbox{\bf reject}}
\newcommand {\THEN}{\mbox{\bf then\ }}
\newcommand {\END}{{\bf end}}
\newcommand {\RETURN}{\mbox{\bf return\ }}
\newcommand {\HALT}{\mbox{\bf halt}}
\newcommand {\REPEAT}{\mbox{\bf repeat\ }}
\newcommand {\UNTIL}{\mbox{\bf until\ }}
\newcommand {\TRUE}{\mbox{\bf true\ }}
\newcommand {\FALSE}{\mbox{\bf false\ }}
\newcommand {\FORALL}{\mbox{\bf for all\ }}
\newcommand {\DOWNTO}{\mbox{\bf down to\ }}


%%%%%%%%    FALL'05
\newcommand{\coursenum}{CS294}
\newcommand{\coursename}{Pseudorandomness and Combinatorial Constructions}
\newcommand{\courseprof}{Professor Luca Trevisan}



%%%%%%%%%%%%%%%%%%%%%%%%%%%%%%%%%%%%%%%%%%%%%%%%%%%%%%%%%%%%%%%%%%%%%%%%%%%
%%%%%%%%%%%%%%%%%%%%%%%%%%%%%%%%%%%%%%%%%%%%%%%%%%%%%%%%%%%%%%%%%%%%%%%%%%%




\newlength{\tpush}
\setlength{\tpush}{2\headheight}
\addtolength{\tpush}{\headsep}


\newcommand{\handout}[3]{\noindent\vspace*{-\tpush}\newline\parbox{\textwidth}
{U.C. Berkeley  \hfill Handout N#1 \newline
\coursenum : \coursename \hfill #2 \newline
\courseprof \hfill #3 \newline
\mbox{}\hrulefill\mbox{}}\vspace*{1ex}\mbox{}\newline
\bigskip
\begin{center}{\Large\bf Notes for Lecture #1}\end{center}
\bigskip}


\newcommand{\problemset}[2]{\noindent\vspace*{-\tpush}\newline\parbox{\textwidth}
{U.C. Berkeley --- \coursenum : \coursename \hfill Handout #1 \newline
\courseprof \hfill #2 \newline
\mbox{}\hrulefill\mbox{}}\vspace*{1ex}\mbox{}\newline
\bigskip
\bigskip}


















\begin{document}

\handout{Lecture 1}{January 17, 2001}{January 21, 2001}{}

This course assumes CS170, or equivalent, as a prerequisite.
We will assume that the reader is familiar with the notions
of algorithm and running time, as well as with basic notions
of discrete math and probability.

A main objective of theoretical computer science is
to understand the amount of resources (time, memory,
communication, randomness , \ldots) needed to solve
computational problems that we care about. While the
design and analysis of algorithms puts upper bounds
on such amounts, computational complexity theory is
mostly concerned with lower bounds; that is we look
for {\em negative results} showing that certain problems
require a lot of time, memory, etc., to be solved.
In particular, we are interested in {\em infeasible}
problems, that is computational problems that require
impossibly large resources to be solved, even on instances
of moderate size. It is very hard to show that a particular
problem is infeasible, and in fact for a lot 
of interesting problems the question of their feasibility
is still open. Another major line of work in complexity
is in understanding the relations between different computational problems
and between different ``modes'' of computation. For
example what is the relative power of algorithms using
randomness and deterministic algorithms, what is the
relation between worst-case and average-case complexity,
how easier can we make an optimization problem if we only
look for approximate solutions, and so on. It is in this
direction that we find the most beautiful, and often
surprising, known results in complexity theory.

Before going any further, let us be more precise in
saying what a computational problem is, and let us
define some important classes of computational problems.
Then we will see a particular incarnation of the notion
of ``reduction,'' the main tool in complexity theory,
and we will introduce \np-completeness, one of the great
success stories of complexity theory.

\section*{Computational Problems}

In a {\em computational problem}, we are given an {\em input}
that, without loss of generality, we assume to be encoded
over the alphabet $\B$, and we want to return in {\em output}
a solution satisfying some property: a computational problem
is then described by the property that the output has to satisfy
given the input.

In this course we will deal with four types of computational
problems: {\em decision} problems, {\em search} problems,
{\em optimization} problems, and {\em counting} problems.\footnote{%
This distinction is useful and natural, but it is also
arbitrary: in fact every problem can be seen as a search problem}
For the moment, we will discuss decision and search problem.

In a {\em decision} problem, given an input $x \in \Bs$, we
are required to give a YES/NO answer. That is, in a decision
problem we are only asked to verify whether the input
satisfies a certain property. An example of decision 
problem is the 3-coloring problem: given an undirected
graph, determine whether there is a way to assign a ``color''
chosen from $\{1,2,3\}$ to each vertex in such a way
that no two adjacent vertices have the same color.

A convenient way to {\em specify} a decision problem
is to give the set $L \subseteq \Bs$ of inputs for which
the answer is YES. A subset of $\Bs$ is also called
a {\em language}, so, with the previous convention,
every decision problem can be specified
using a language (and every language specifies a decision
problem). For example, if we call 3COL the subset of
$\Bs$ containing (descriptions of) 3-colorable graphs,
then 3COL is the language that specifies the 3-coloring
problem. From now on, we will talk about decision problems
and languages interchangeably.

In a {\em search} problem, given an input $x\in \Bs$ we
want to compute some answer $y\in \Bs$ that is in some
relation to $x$, if such a $y$ exists. Thus, a search
problem is specified by a relation $R \subseteq \Bs \times \Bs$,
where $(x,y) \in R$ if and only if $y$ is an admissible
answer given $x$.

Consider for example the search version of the 3-coloring
problem: here given an undirected graph $G= (V,E)$ we
want to find, if it exists, a coloring $c : V \to \{1,2,3\}$
of the vertices, such that for every $(u,v)\in V$ we
have $c(u) \neq c(v)$. This is different (and more demanding)
than the decision version, because beyond being asked
to determine whether such a $c$ exists, we are also asked
to construct it, if it exists. Formally, the 3-coloring
problem is specified by the relation $R_{\rm 3COL}$ that
contains all the pairs $(G,c)$ where $G$ is a 3-colorable
graph and $c$ is a valid 3-coloring of $G$.

\section*{P and NP}

In most of this course, we will study the {\em asymptotic} complexity
of problems. Instead of considering, say, the time required to
solve 3-coloring on graphs with $10,000$ nodes on some particular
model of computation, we will ask what is the best asymptotic
running time of an algorithm that solves 3-coloring on all
instances. In fact, we will be much less ambitious, and we will
just ask whether there is a ``feasible'' asymptotic algorithm
for 3-coloring. Here feasible refers more to the rate of growth
than to the running time of specific instances of reasonable size.

A standard convention is to call an algorithm ``feasible'' if
it runs in polynomial time, i.e. if there is some polynomial $p$
such that the algorithm runs in time at most $p(n)$ on inputs of
length $n$. 

We denote by $\p$ the class of decision problems that are
solvable in polynomial time.

We say that a search problem defined by a relation $R$ is a
NP search problem if the relation is efficiently computable
and such that solutions, if they exist, are short. Formally,
$R$ is an $\np$ search problem if there is a polynomial time
algorithm that, given $x$ and $y$, decides whether $(x,y)\in R$,
and if there is a polynomial $p$ such that if $(x,y)\in R$ then
$|y| \leq p(|x|)$.

We say that a decision problem $L$ is an NP decision problem
if there is some NP relation $R$ such that $x\in L$ if and
only if there is a $y$ such that $(x,y) \in R$. Equivalently,
a decision problem $L$ is an NP decision problem if
there is a polynomial time algorithm $V(\cdot,\cdot)$
and a polynomial $p$ such that $x\in L$ if and only if
there is a $y$, $|y| \leq p(|x|)$ such that $V(x,y)$ accepts.

We denote by $\np$ the class of NP decision problems.

\section*{Reductions}

Let $A$ and $B$ be two decision problems. We say that $A$ reduces
to $B$, denoted $A \leq B$, if there is a polynomial time computable
function $f$ such that $x\in A$ if and only if $f(x) in B$.

Two immediate observations: if $A\leq B$ and $B$ is in $\p$, 
then also $A\in \p$ (conversely, if $A\leq B$, and $A\not\in \p$
then also $B\not\in \p$); if $A\leq B$ and $B\leq C$, then
also $A\leq C$.

\section*{NP-completeness}

A decision problem $A$ is $\np$-hard if for every 
problem $L \in \np$ we have $L \leq A$. A decision
problem $A$ is $\np$-complete if it is $\np$-complete
and it belongs to \np.

It is a simple observation that if $A$ is \np-complete,
then $A$ is solvable in polynomial time if and only if
$\p = \np$.

\section*{An NP-complete problem}

Consider the following decision problem, that we call $U$: we are given
in input $(M,x,t,l)$ where $M$ is a Turing machine,
$x\in \Bs$ is a possible input, and $t$ and $l$ are integers
encoded in unary\footnote{The ``unary'' encoding of an integer
$n$ is a sequence of $n$ ones.}, and the problem is
to determine whether there is a $y \in \B^*$, $|y| \leq l$,  such that
$M(x,y)$ accepts in $\leq t$ steps.

It is immediate to see that $U$ is in \np. One can define
a procedure $V_U$ that on input $(M,x,t,l)$ and $y$ accepts
if and only if $|y| \leq l$, and $M(x,y)$ accepts in at most $t$
steps.

Let $L$ be an NP decision problem. Then there are algorithm $V_L$,
and polynomials $T_L$ and $p_L$, such that $x\in L$ if and only
if there is $y$, $|y| \leq p_L(|x|)$ such that $V_L(x,y)$ accepts;
furthermore $V_L$ runs in time at most $T_L(|x| + |y|)$. We give
a reduction from $L$ to $U$. The reduction maps $x$
into the instance $f(x) = (V_L,x,T_L(|x|+p_L(|x|)),p_L(|x))$. Just by
applying the definitions, we can see that $x\in L$ if and
only $f(x) \in U$.

\end {document}