\documentclass[11pt]{article}
\usepackage
[pdftex,pagebackref,letterpaper=true,colorlinks=true,pdfpagemode=none,urlcolor=blue,linkcolor=blue,citecolor=blue,pdfstartview=FitH]{hyperref}

\usepackage{amsmath,amsfonts}
\usepackage{graphicx}

\bibliographystyle{alpha}

\setlength{\oddsidemargin}{0pt}
\setlength{\evensidemargin}{0pt}
\setlength{\textwidth}{6.0in}
\setlength{\topmargin}{0in}
\setlength{\textheight}{8.5in}

\setlength{\parindent}{0in}
\setlength{\parskip}{5pt}


%%%%%%%%% For wordpress conversion

\def\more{}

\usepackage{ulem}
\def\em{\it}
\let\emph=\it



%Formatting


\setlength{\topmargin}{0in}
\setlength{\oddsidemargin}{0 in}
\setlength{\evensidemargin}{0 in}
\setlength{\textwidth}{6.5 in}
\setlength{\textheight}{8.5 in}
\setlength{\headsep}{0.75 in}
\setlength{\parindent}{0 in}
\setlength{\parskip}{0.05 in}

%Theorems, proofs, etc.

\newtheorem{theorem}{Theorem}
\newtheorem{lemma}[theorem]{Lemma}
\newtheorem{corollary}[theorem]{Corollary}
\newtheorem{proposition}[theorem]{Proposition}
\newtheorem{claim}[theorem]{Claim}
\newtheorem{exercise}{Exercise}
\newtheorem{conjecture}{Conjecture}
\newtheorem{example}{Example}
\newtheorem{remark}{Remark}
\newtheorem{definition}[theorem]{Definition}
\newenvironment{proof}{\noindent {\sc Proof:}}{$\Box$\medskip} 
\newtheorem{question}{Open Question}
\newtheorem{problem}{Problem}



%% Macros for complexity theory notation
%% Complexity classes
\newcommand\p{\mbox{\bf P}}
\newcommand\np{\mbox{\bf NP}}
\newcommand\am{\mbox{\bf AM}}
\newcommand\ma{\mbox{\bf MA}}
\newcommand\distnp{\mbox{\bf DistNP}}
\newcommand\cnp{\mbox{\bf coNP}}
\newcommand\corp{\mbox{\bf coRP}}
\newcommand\sigmatwo{\mbox{\bf $\Sigma_2$}}
\newcommand\sigmathree{\mbox{\bf $\Sigma_3$}}
\newcommand\pitwo{\mbox{\bf $\Pi_2$}}
\newcommand\rp{\mbox{\bf RP}}
\newcommand\zpp{\mbox{\bf ZPP}}
\newcommand\bpp{\mbox{\bf BPP}}
\newcommand\ph{\mbox{\bf PH}}
\newcommand\pspace{\mbox{\bf PSPACE}}
\newcommand\npspace{\mbox{\bf NPSPACE}}
\newcommand\aczero{\mbox{\bf AC$^0$}}
\newcommand\dl{\mbox{\bf L}}
\newcommand\nl{\mbox{\bf NL}}
\newcommand\rl{\mbox{\bf RL}}
\newcommand\sll{\mbox{\bf SL}}
\newcommand\bpl{\mbox{\bf BPL}}
\newcommand\conl{\mbox{\bf coNL}}
\newcommand\sharpp{\mbox{\#{\bf P}}}
\newcommand\parityp{\mbox{$\oplus$ {\bf P}}}
\newcommand\ip{\mbox{\bf IP}}
\newcommand\pcp{\mbox{\bf PCP}}
\newcommand\dtime{\mbox{\bf DTIME}}
\newcommand\ntime{\mbox{\bf NTIME}}
\newcommand\bptime{\mbox{\bf BPTIME}}
\newcommand\dspace{\mbox{\bf SPACE}}
\newcommand\nspace{\mbox{\bf NSPACE}}
\newcommand\cnspace{\mbox{\bf coNSPACE}}
\newcommand\exptime{\mbox{\bf EXPTIME}}
\def\exp{\mbox{\bf EXP}}
\def\ep{\mbox{\bf E}}
\newcommand\nexptime{\mbox{\bf NEXPTIME}}
\newcommand\genclass{\mbox{$\cal C$}}
\newcommand\cogenclass{\mbox{\bf co$\cal C$}}
\newcommand\size{\mbox{\bf SIZE}}

\def\phi{\varphi}

%%Computational problems

\newcommand\sat{\mbox{SAT}}
\newcommand\tsat{\mbox{3SAT}}
\newcommand\tqbf{\mbox{TQBF}}
\newcommand\permp{\mbox{\sc Permanent}}
\def\perm{\operatorname{perm}}

%% Notation for integers, natural numbers, reals, fractions, sets, cardinalities
%%and so on

\newcommand\Z{\mbox{$\mathbb Z$}}
\newcommand\N{\mbox{$\mathbb N$}}
\newcommand\R{\mbox{$\mathbb R$}}
\newcommand\F{\mbox{$\mathcal{F}$}}
\newcommand\B{\{0,1\}}      % boolean alphabet  use in math mode
\newcommand\Bs{\{0,1\}^*}   % B star            use in math mode
\def\H{\{-1,1\}}      
\def\S{{\cal S}}

\newcommand{\bzero}{{\bf 0}}


\newcommand\true{\mbox{\sc True}}
\newcommand\false{\mbox{\sc False}}

\newcommand{\marginlabel}[1]%
{\mbox{}\marginpar{\it{\raggedleft\hspace{0pt}#1}}}
\newcommand\poly{\mbox{poly}}  %usage \poly(n)
\newcommand{\floor}[1]{\lfloor\, $#1$\,\rfloor}
\newcommand{\ceil}[1]{\lceil\, $#1$\,\rceil}
\newcommand{\comp}[1]{\overline{#1}}


%\newcommand\to{\rightarrow} 
\newcommand\xor{\oplus}
\newcommand\bigxor{\bigoplus}
\newcommand{\logred}{\leq_{\log}}
\def\implies{\Rightarrow}


%% probability stuff

\newcommand\E{\mathop\mathbb{E}}
\newcommand\pr{\mathop{\bf {Pr}}}
\newcommand\var{\mathop{\bf Var}}


%% macros to write pseudo-code

\newlength{\pgmtab}  %  \pgmtab is the width of each tab in the
\setlength{\pgmtab}{1em}  %  program environment
 \newenvironment{program}{\renewcommand{\baselinestretch}{1}%
\begin{tabbing}\hspace{0em}\=\hspace{0em}\=%
\hspace{\pgmtab}\=\hspace{\pgmtab}\=\hspace{\pgmtab}\=\hspace{\pgmtab}\=%
\hspace{\pgmtab}\=\hspace{\pgmtab}\=\hspace{\pgmtab}\=\hspace{\pgmtab}\=%
\+\+\kill}{\end{tabbing}\renewcommand{\baselinestretch}{\intl}}
\newcommand {\BEGIN}{{\bf begin\ }}
\newcommand {\ELSE}{{\bf else\ }}
\newcommand {\IF}{{\bf if\ }}
\newcommand {\FOR}{{\bf for\ }}
\newcommand {\TO}{{\bf to\ }}
\newcommand {\DO}{{\bf do\ }}
\newcommand {\WHILE}{{\bf while\ }}
\newcommand {\ACCEPT}{{\bf accept}}
\newcommand {\REJECT}{\mbox{\bf reject}}
\newcommand {\THEN}{\mbox{\bf then\ }}
\newcommand {\END}{{\bf end}}
\newcommand {\RETURN}{\mbox{\bf return\ }}
\newcommand {\HALT}{\mbox{\bf halt}}
\newcommand {\REPEAT}{\mbox{\bf repeat\ }}
\newcommand {\UNTIL}{\mbox{\bf until\ }}
\newcommand {\TRUE}{\mbox{\bf true\ }}
\newcommand {\FALSE}{\mbox{\bf false\ }}
\newcommand {\FORALL}{\mbox{\bf for all\ }}
\newcommand {\DOWNTO}{\mbox{\bf down to\ }}


%%%%%%%%    FALL'05
\newcommand{\coursenum}{CS294}
\newcommand{\coursename}{Pseudorandomness and Combinatorial Constructions}
\newcommand{\courseprof}{Professor Luca Trevisan}



%%%%%%%%%%%%%%%%%%%%%%%%%%%%%%%%%%%%%%%%%%%%%%%%%%%%%%%%%%%%%%%%%%%%%%%%%%%
%%%%%%%%%%%%%%%%%%%%%%%%%%%%%%%%%%%%%%%%%%%%%%%%%%%%%%%%%%%%%%%%%%%%%%%%%%%




\newlength{\tpush}
\setlength{\tpush}{2\headheight}
\addtolength{\tpush}{\headsep}


\newcommand{\handout}[3]{\noindent\vspace*{-\tpush}\newline\parbox{\textwidth}
{U.C. Berkeley  \hfill Handout N#1 \newline
\coursenum : \coursename \hfill #2 \newline
\courseprof \hfill #3 \newline
\mbox{}\hrulefill\mbox{}}\vspace*{1ex}\mbox{}\newline
\bigskip
\begin{center}{\Large\bf Notes for Lecture #1}\end{center}
\bigskip}


\newcommand{\problemset}[2]{\noindent\vspace*{-\tpush}\newline\parbox{\textwidth}
{U.C. Berkeley --- \coursenum : \coursename \hfill Handout #1 \newline
\courseprof \hfill #2 \newline
\mbox{}\hrulefill\mbox{}}\vspace*{1ex}\mbox{}\newline
\bigskip
\bigskip}


















\begin{document}

\handout{Lecture 5}{January 31, 2001}{\today}{Scribe: Chris Harrelson}

Today we will prove that $\nl = \conl$ and discuss the polynomial hierarchy.

\section{$\nl= \conl$}

In order to prove that these two classes are the same, we will show
that there is an $\nl$ Turing machine which solves
$\overline{\textsc{stconn}}$.  $\overline{\textsc{stconn}}$ is the
problem of deciding, given a directed graph $G$, together with special
vertices $s$ and $t$, whether $t$ is \emph{not} reachable from $s$.
Note that $\overline{\textsc{stconn}}$ is $\conl$-complete.

Once we have the machine, we know that $\conl \subseteq
\nl$, since any language A in $\conl$ can be reduced to
$\overline{\textsc{stconn}}$, and since $\overline{\textsc{stconn}}$
has been shown to be in $\nl$ (by the existence of our machine), so is A.
Also, $\nl \subseteq \conl$, since if $\overline{\textsc{stconn}} \in \nl$, by
definition $\textsc{stconn} \in \conl$, and since
\textsc{stconn} is $\nl$-complete, this means that any problem in $\nl$ can
be reduced to it and so is also in $\conl$.  Hence $\nl = \conl$.

\subsection{A simpler problem first}

Now all that remains to be shown is that this Turing machine exists.
First we will solve a simpler problem than $\overline{\textsc{stconn}}$.  We will assume
that in addition to the usual inputs $G$, $s$ and $t$, we also have an input $r$, which
we will assume is equal to the number of vertices reachable from $s$ in $G$, including $s$.

Given these inputs, we will construct a non-deterministic Turing machine which decides whether $t$
is reachable from $s$ by looking at all subsets of $r$ vertices in $G$, halting with \textsc{yes}
if it sees a subset of vertices which are all reachable from $s$ but do not include $t$, and
halting with \textsc{no} otherwise.  Here is the algorithm:

\begin{program}
input: $G = (V,E)$, $s$, $t$, $r$ \\
output: \textsc{yes} if it discovers that $t$ is not reachable from $s$, and \textsc{no} otherwise \\
assumption: there are exactly $r$ distinct vertices reachable from $s$ \\
\\
$c$ := 0 \\
\FORALL \= $v \in (V - \{t\})$ \DO \\
        \> non-deterministically guess if $v$ is reachable from $s$ in $k$ steps \\
        \> \IF \= guess = YES \THEN \\
	\>     \> $p$ := $s$ \\
    	\>     \> \FOR \= $i$ := 1 to $k$ \DO \\
        \>     \>      \> non-deterministically pick a neighbor $q$ of $p$ \\
        \>     \>      \> p := q \\
        \>     \> if $p$ is not equal to $v$, reject \\
        \>     \> $c$ := $c + 1$ \\
\\
\IF $c = r$ \THEN return \textsc{yes}, otherwise return \textsc{no} \\
\end{program}
Exercise: verify that this algorithm is indeed in \nl.

Notice that in the algorithm above, $c$ can only be incremented for a
vertex $v$ that is actually reachable from $s$.  Since there are
assumed to be exactly $r$ such vertices, $c$ can be at most $r$ at the
end of the algorithm, and if it is exactly $r$, that means that there
are $r$ vertices other than $t$ which are reachable from $s$, meaning
that $t$ by assumption cannot be reachable form $s$.  Hence the
algorithm accepts if and only if it discovers that $t$ is not
reachable from $s$.

\subsection{Finding $r$}

Now we need to provide an \nl-algorithm that finds $r$.  Let's first try this algorithm:

\begin{program}
input: $G = (V,E)$, $s$ \\
output: the number of vertices reachable from $s$ (including $s$ in this count) \\
\\
$c$ := 0 \\
\FORALL \= $v \in V$ \DO \\
        \> non-deterministically guess if $v$ is reachable from $s$ in $k$ steps \\
        \> \IF \= guess = \textsc{yes} \THEN \\
        \>     \> $p$ := $s$ \\
        \>     \> \FOR \= $i$ := 1 to $k$ \DO \\
        \>     \>      \> non-deterministically guess a neighbor $q$ of $p$ (possibly not moving at all) \\
        \>     \>      \> $p$ := $q$ \\
        \>     \> if $p \ne v$ reject \\
        \>     \> $c$ := $c + 1$ \\
\\
return $c$
\end{program}

\textbf{This algorithm has a problem}.  It will only return a number
$c$ which is at most $r$, but we need it to return \emph{exactly} $r$.  We
need a way to force it to find all vertices which are reachable from
$s$.  Towards this goal, let's define $r_i$ to be the set of vertices reachable from $s$
in at most $i$ steps.  Then $r = r_{n-1}$, where $n$ is the number of
vertices in $G$.  The idea is to try to compute $r_i$ from $r_{i-1}$ and
repeat the procedure $n-2$ times. Now here is another try at an algorithm:

\begin{program}
input: $G = (V,E)$, $s$, $i$, $r_{i-1}$ \\
output: the number of vertices reachable from $s$ in at most $i$ steps (including $s$ in this count) \\
assumption: $r_{i-1}$ is the exact number of vertices reachable from $s$ in at most $i-1$ steps \\
\\
$c$ := 0 \\
\FORALL \= $v \in V$ \DO \\
        \> $d$ := 0 \\
        \> \FORALL \= $w \in V$ \DO \\
        \>         \> non-deterministically guess if $w$ is reachable from $s$ in at most $k-1$ steps \\
        \>         \> \IF \= guess = \textsc{yes} \THEN \\
        \>         \>     \> $p$ := $s$ \\
        \>         \>     \> \FOR \= $i$ := $1$ to $k-1$ \DO \\
        \>         \>     \>      \> non-deterministically pick a neighbor $q$ of $p$ (possibly not moving at all) \\
        \>         \>     \>      \> $p$ := $q$ \\
        \>         \>     \> \IF $p \ne v$ \THEN reject \\
        \>         \>     \> $d$ := $d + 1$ \\
        \>         \>     \> \IF \= $v$ is a neighbor of $w$, or if $v = w$ \THEN \\
        \>         \>     \>     \> $c$ := $c + 1$ \\
        \>         \>     \>     \> break out of the inner loop and start the next iteration of the outer loop \\
        \>         \IF $d < r_{i-1}$ reject \\
\\
return $c$
\end{program}

Here is the idea behind the algorithm: for each vertex $v$, we need to
determine if it is reachable from $s$ in at most $i$ steps.  To do
this, we can loop over all vertices which are a distance at most $i-1$
from $s$, checking to see if $v$ is either equal to one of these vertices
or is a neighbor of one of them (in which case it would be reachable in
exactly $i$ steps).  The algorithm is able to force all vertices of distance
at most $i-1$ to be considered because it is given $r_{i-1}$ as an input.

Now, putting this algorithm together with the first one listed above,
we have shown that $\overline{\textsc{stconn}} \in \nl$, implying that
$\nl = \conl$.  In fact, the proof can be generalized to show that if
a decision problem $A$ is solvable in non-deterministic space $s(n) =
\Omega(\log n)$, then $\overline{A}$ is solvable in non-deterministic
space $O(s(n))$.

\section{The polynomial hierarchy}

One way to look at the difference between $\np$ and $\cnp$ is that a
decision problem in $\np$ is asking a sort of ``does there exist''
question, where the existence of the answer can by definition be
efficiently represented.  On the other hand, $\cnp$ asks ``is it
true for all'' questions, which do not seem to have simple, efficient
proofs.  In this way, complexity theorists think of there being a
$\exists$ associated with $\np$ and a $\forall$ associated with $\cnp$.

\subsection{Stacks of quantifiers}

Now suppose you had a decision problem $A$ which asked something of
the following form:
\[ x \in A \Leftrightarrow \exists~y_1~\mbox{s.t.}~|y_1| \le p(|x|)~\forall~y_2~\mbox{s.t.}~|y_2| \le p(|x|)~V(x,y_1, y_2) \]
In other words, a Turing machine solving problem $A$ should return
\textsc{yes} on an input $x$ if an only if there exists some string
$y_1$ such that for all strings $y_2$ (both of polynomial length), the
predicate $V(x,y_1, y_2)$ holds.  An example of such a problem is
this: given a Boolean formula $\phi$ over variables $x_1, \ldots,
x_n$, is there a formula $\phi'$ which is equivalent to $\phi$ and is
of size at most $k$?  In this case, $y_1$ is the formula $\phi'$,
$y_2$ is an arbitrary assignment to the variables $x_1, \ldots, x_n$,
and $V(x,y_1, y_2)$ is the predicate which is true if and only if
$x[y_2]$ and $y_1[y_2]$ are both true or both false, meaning that
under the variable assignment $y_2$, $\phi$ and $\phi'$ agree.  Notice
that $\phi'$ is equivalent to $\phi$ if and only if it agrees with
$\phi$ under all assignments of Boolean values to the variables.

As we will see, the problem $A$ is a member of the class $\Sigma_2$ in
the second level of the polynomial hierarchy.

\subsection{The hierarchy}

The polynomial hierarchy starts with familiar classes on level one: $\Sigma_1 = \np$
and $\Pi_1 = \cnp$.  For all $i \ge 1$, it includes two classes, $\Sigma_i$ and $\Pi_i$,
which are defined as follows:
\[ A \in \Sigma_i \Leftrightarrow \exists y_1.~\forall y_2.~\ldots~.Q y_i.~V_A(x,y_1, \ldots, y_i) \]
and
\[ B \in \Pi_i \Leftrightarrow \forall y_1.~\exists y_2.~\ldots~.Q' y_i.~ V_B(x,y_1, \ldots, y_i) \]
where the predicates $V_A$ and $V_B$ depend on the problems $A$ and $B$, and
$Q$ and $Q'$ represent the appropriate quantifiers, which depend on whether $i$
is even or odd (for example, if $i=10$ then the quantifier $Q$ for $\Sigma_{10}$ is
$\forall$, and the quantifier $Q'$ for $\Pi_{10}$ is $\exists$).  For clarity, we have also
omitted the $p(\cdot)$ side conditions, but they are still there.

One thing that is easy to see is that $\Pi_k = \mbox{co}\Sigma_k$.  Also, note that, for all
$i \le k-1$, $\Pi_i \subseteq \Sigma_k$ and $\Sigma_i \subseteq \Sigma_k$.  These subset relations
hold for $\Pi_k$ as well.  This can be seen by noticing that the predicates $V$ do not need
to ``pay attention to'' all of their arguments, and so can represent classes lower on the
hierarchy which have a smaller number of them.

Here are some more facts about the polynomial hierarchy (proof left to the reader):

\begin{enumerate}
\item $\Pi_i$ and $\Sigma_i$ have complete problems for all $i$.
\item A $\Sigma_i$-complete problem is not in $\Pi_j$, $j \le i-1$, unless $\Pi_j = \Sigma_i$,
and it is not in $\Sigma_j$ unless $\Sigma_j = \Sigma_i$.
\item Suppose that $\Sigma_i = \Pi_i$ for some $i$.  Then $\Sigma_j = \Pi_j = \Sigma_i = \Pi_i$ for
all $j \ge i$.
\item Suppose that $\Sigma_i = \Sigma_{i+1}$ for some $i$.  Then $\Sigma_j = \Pi_j = \Sigma_i$ for
all $j \ge i$.
\item Suppose that $\Pi_i = \Pi_{i+1}$ for some $i$.  then $\Sigma_j = \Pi_j = \Pi_i$ for all
$j \ge i$.
\end{enumerate}

While it seems like an artificial construction right now, in future lectures we will see that
the polynomial hierarchy helps us to understand other complexity classes.

\subsection{An alternate characterization}

The polynomial hierarchy can also be characterized in terms of
``oracle machines.''  The idea here is that, instead of a standard
Turing machine, we consider one which is augmented with an oracle of a
certain power which can be consulted as many times as desired, and
using only one computational step each time.  Syntactically, this can
be written as follows:

Let $A$ be some decision problem and $\mathcal{M}$ be a class of Turing
machines.  Then $\mathcal{M}^A$ is defined to be the class of machines obtained
from $\mathcal{M}$ by allowing instances of $A$ to be solved in one step.
For example, $\np^{\tsat} = \Sigma_2$.

In general, let $L$ be some $\Sigma_{i-1}$-complete problem.  Then it can be verified
that $\Sigma_i = \np^L$ and $\Pi_i = \cnp^L$.

\end{document}


